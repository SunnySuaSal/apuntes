\documentclass[a4paper, 12pt]{article}

%Paquetes recomendados
\usepackage[utf8]{inputenc} %Para acentos en español
\usepackage[spanish]{babel} %Idioma español
\usepackage{amsmath} %Matemáticas avanzadas
\usepackage{amsfonts} %Conjuntos numéricos
\usepackage{amssymb} %símbolos extra
\usepackage{geometry} %Control de márgenes
\usepackage{physics} %Notación elegante de derivadas (opcional)
\usepackage{hyperref} %Para enlaces e índice interactivo

\geomety{top=2.5cm, bottom=2.5cm, left=3cm, right=3cm}

%Comandos personalizados para mejorar presentación
\newcommand{\orden}[1]{\textbf{Orden:} #1}
\newcommand{\grado}[1]{\textbf{Grado:} #1}
\newcommand{\solucion}{\textbf{Solución:} }

\begin{document}
  %Portada
  \begin{titlepage}
    \centering
    \vspace*{4cm}
    {\large \textbf{Ecuaciones Diferenciales} \par}
    \vspace{1.5cm}
    {\large Curso de Cálculo D \par}
    \vspace{2cm}
    {\large Elaborado por: Sunny Saldaña \par}
    \vspace{1.5cm}
    {\large Fecha: \today \par}
    \vfill
  \end{titlepage}

  %índice autimático
  \tableofcontents
  \newpage

  %Sección de introducción
  \section{Introducción}
  En este documento se estudian las ecuaciones diferenciales, sus tipos, orden, grado y métodos de solución.

  %Sección de definiciones
  \section{Definiciones Básicas}
  \subsection{Orden de una derivada}
  El \textbf{orden} de una derivada indica cuántas veces se ha derivado la función. Ejemplos:
  \begin{itemize}
    \item $\dv{y}{x}$ es una derivada de orden 1.
    \item $\dv[2]{y}{x}$ es una derivada de orden 2.
    \item $\dv[n]{y}{x}$ es una derivada de orden $n$.
  \end{itemize}

  \subsection{Orden de una Ecuación Diferencial}
  El \textbf{orden de una ecuación diferencial} es el orden de la derivada de mayor nivel que aparece en la ecuación.

  \subsection{Grado de una Ecuación Diferencial}
  El \textbf{grado de una ecuación diferencial} es el exponente de la derivada de mayor orden, siempre que la ecuación esté despejada
  y sea polinómica en las derivadas.

  %Sección de ejemplos
  \section{Ejemplos de Identificación de Orden y Grado}
  \subsection{Ejemplo 1}
  \[
  \frac{d^2y}{dx^2} + \frac{dy}{dx} = e^x
  \]
  \orden{2} \quad \grado{1}

  \subsection{Ejemplo 2}
  \[ 
  \left( \frac{dy}{dx} \right)^3 + y = \sin(x)
  \]
  \orden{1} \quad \grado{3}

  \subsection{Ejemplo 3}
  \[ 
  \sqrt{\frac{d^2y}{dx^2} + \frac{dy}{dx} = x^2}
  \]
  \orden{2} \quad \grado{No definido} (No es polinómica en la derivada)

  %Sección de procedimientos
  \section{Resolución de Ecuaciones Diferenciales}
  \subsection{Ejemplo resuelto paso a paso}
  
  \ejemplo Reslver la ecuación diferencial:
  \[ 
  \dv{y}{x} + y = e^x
  \]

  \solucion

  \begin{align*}
    \dv{y}{x} + y &= e^x \\
    \text{Factor integrante: } \mu(x) &= e^{\int 1 \, dx} = e^{x} \\
    \text{Multiplicamos toda la ecuación por } e^x: \quad & e^x \dv{y}{x} + e^x y = e^{2x} \\
    \text{Lado izquierdo es derivada del producto:} \quad & \dv{}{x} \left( e^x y \right) = e^{2x} \\
    \text{Integramos:} \quad & e^x y = \int e^{2x} dx = \frac{e^{2x}}{2} + C \\
    \text{Despejamos y:} \quad & y(x) = \frac{e^{2x}}{2e^x} + \frac{C}{e^x} = \frac{e^{x}}{2} + C e^{-x}
  \end{align*}

  \newpage

  %Sección final
  \section{Conlusión}
  Este documento presenta los conceptos básicos y ejemplos escenciales para entender el orden, grado
  y resolución de ecuaciones diferenciales de primer y segundo orden.

\end{document}
