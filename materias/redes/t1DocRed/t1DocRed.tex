\documentclass[12pt]{article}

\usepackage[utf8]{inputenc}
\usepackage{amsmath, amssymb, amsfonts, cancel}
\usepackage{graphicx}
\usepackage[letterpaper, left=1in, right=1in, top=1in, bottom=1in]{geometry}
\usepackage{booktabs}
\usepackage{multirow}
\usepackage{float}
\usepackage{caption}
\usepackage{colortbl}
\usepackage{xcolor}
\definecolor{paleYellow}{RGB}{255,255,180}
\usepackage{physics}
\usepackage{hyperref}
\usepackage[spanish]{babel}

\title{
Tarea 1: Documentación de red\\[0.5em]
\large Administración de redes
}

\author{
Jorge Alberto Suarez Saldana\\
\small Matrícula: 355992\\
\small Profesor: Raymundo Antonio González Grimaldo
}

\date{\today}

\begin{document}
\maketitle

\section{Topología de la red}
Podemos observar en la Figura \ref{fig:topologia} un diagrama de la topología de la red en el modo lógico del simulador
Cisco Packet Tracer.
\begin{figure}[H]
  \centering
  \includegraphics[width=0.8\textwidth]{diagramaTopologia.png}
  \caption{Topología de la red}\label{fig:topologia}
\end{figure}

\section{Generación de subredes}
\subsection{IPv4}
Una IPv4 es una dirección (identificador numérico) de la forma \textbf{x.x.x.x} donde cada x es 
un \textit{octeto}, esto porque está compuesto de 8 bits, alcanzando entonces cada octeto el rango 
decimal de 0 - 255.

\subsection{Prefijo de red}
Al existir una inmensa cantidad de direcciones IP posibles y surgir la necesidad de dividir redes grandes 
en subredes más pequeñas, se inventó el \textit{prefijo de red}, que utilizando una notación de diagonal (/n) 
divide el rango de direcciones IP en dos: la porción de red y la porción de host, comenzando la división en el bit n.

Por ejemplo, en este ejercicio, tenemos la red principal \textbf{192.168.128.0/20} esto quiere decir que, 
hasta el bit número 20 de la dirección IP correspnde a la porción de red, es decir, aquellos bits que se mantienen 
estáticos para todas las direcciones que asignemos dentro de esta red, por lo que funcionan como un indicador de la misma.
El resto de bits después del bit 20 en este ejemplo corresponden a la porción de host, es decir, aquellos bits que 
sí cambian para darle a cada host (end device) una dirección IP única dentro de la red.

Continuando con el ejemplo, el bit 20 queda dentro del tercer octeto, por lo que para calcular 
las subredes, podemos expresar la red como se muestra en la Figura \ref{fig:redBinarioExp}.

\begin{figure}[H]
  \begin{center}
    \includegraphics[width=0.8\textwidth]{redBinarioExp.jpeg}
  \end{center}
  \caption{Red con el tercer y cuarto octeto en binario}\label{fig:redBinarioExp}
\end{figure}

\subsection{Subneteo}
Ahora, en el ejercicio nos piden dividir esta red en 7 subredes (delimitadas por los routers).
Para calcular el prefijo de red de estas subredes (asignandole a cada red la máxima cantidad de direcciones posibles 
de manera equivalente), calculamos el numero de bits que requerimos para 7 subredes.

\[ 2^3 = 8 \]

Con 3 bits, podemos definir hasta 8 subredes, lo más cercano a lo que nos pide el ejercicio sin pasarnos.
Entonces, el prefijo de red para cada una de estas 7 subredes será \[ 20 + 3 = 23 \rightarrow /23 \]

\subsection{Máscara de red}
Para calcular la máscara de red (en decimal, como normalmente se pide al configurar las redes) a partir 
del prefijo de red, ponemos todos los primeros n bits (porción de red) en 1 y el resto (porción de host) 
en 0 y convertimos a decimal.

En este ejercicio, con el prefijo de red $/23$, tenemos (Figura \ref{fig:mascaraRedExp}):

\begin{figure}[H]
  \begin{center}
    \includegraphics[width=0.8\textwidth]{mascaraRedExp.jpeg}
  \end{center}
  \caption{Máscara de red en binario y decimal}\label{fig:mascaraRedExp}
\end{figure}

Tenemos entonces la máscara de red: \textbf{255.255.254.0}

\subsection{Tamaño del bloque}
Para subneteo en decimal, se utiliza la fórmula:
\[ \text{Tamaño del bloque} = 256 - \text{Valor del octeto de la máscara} \]
En este caso: $256 - 254 = 2$. Ese 2 es el salto decimal entre subredes en ese octeto.

Así, podemos calcular nuestras 7 subredes:
\begin{enumerate}
  \item 192.168.128.0/23
  \item 192.168.130.0/23
  \item 192.168.132.0/23
  \item 192.168.134.0/23
  \item 192.168.136.0/23
  \item 192.168.138.0/23
  \item 192.168.140.0/23
\end{enumerate}

\subsection{Reglas generales para subredes}
Para cualquier IPv4:
\begin{itemize}
  \item ID de red:
    Es la primera dirección del bloque. Todos los bits de host en 0.
  \item Primera dirección utilizable:
    ID de red $+ 1$
  \item Broadcast:
    Es la última dirección del bloque. Todos los bits de host en 1.
  \item Última dirección utilizable:
    Broadcast $- 1$
  \item Máscara de red:
    La misma para todas.
\end{itemize}

\subsection{Convertir de decimal a binario}
Justo antes de calcular las subredes para el ejercicio, resulta útil 
repasar cómo convertir npumeros decimales a binarios.

Primero necesitamos saber cuántos bits necesitamos para representar el número, con la siguiente 
fórmula: \[ n = \left\lceil \log_2(d) \right\rceil \] donde n es el número de bits necesarios 
para representar el número decimal d.

Pongamos de ejemplo el número decimal 130: \[ \log_2(130) = 7.0223 \rightarrow \lceil 7.0223 \rceil = 8 \]

Entonces Escribimos los valores de $2^0 \to 2^{n-1}$ en orden ascendente, como en la Figura \ref{fig:twoPowersEmpty}.

\begin{figure}[H]
  \begin{center}
    \includegraphics[width=0.8\textwidth]{twoPowersEmpty.jpeg}
  \end{center}
  \caption{Potencias de dos ordenadas}\label{fig:twoPowersEmpty}
\end{figure}

Y el algritmo es:
\begin{enumerate}
  \item Busca la potencia mayor que no pase de $d$ (y asígnale 1).
  \item Resta el valor de esa potenica a $d$.
  \item Repite el paso 1 hasta que $d = 0$ (asigna 0 a las potencias restantes).
\end{enumerate}

Continuando con el ejemplo de 130 (Figura \ref{fig:binarioADecimal130}):

\begin{figure}[H]
  \begin{center}
    \includegraphics[width=0.8\textwidth]{binarioADecimal130.jpeg}
  \end{center}
  \caption{Ejemplo conversion 130 decimal a binario}\label{fig:binarioADecimal130}
\end{figure}

Así, tenemos que $130_{10} = 10000010_2$

\subsection{Terminando información de subredes del ejercicio}
Podemos entonces calcular la información de las 7 subredes para este ejercicio:

\begin{table}[H]
\centering
\caption*{Subred 1: 192.168.128.0/23}
\begin{tabular}{@{}ll@{}}
\toprule
ID de red                   & 192.168.128.0 \\ 
Primera dirección utilizable & 192.168.128.1 \\
Última dirección utilizable  & 192.168.128.254 \\
Broadcast                   & 192.168.128.255 \\
Máscara de red               & 255.255.254.0 \\
\bottomrule
\end{tabular}
\end{table}


\begin{table}[H]
\centering
\caption*{Subred 2: 192.168.130.0/23}
\begin{tabular}{@{}ll@{}}
\toprule
ID de red                   & 192.168.130.0 \\ 
Primera dirección utilizable & 192.168.130.1 \\
Última dirección utilizable  & 192.168.131.254 \\
Broadcast                   & 192.168.131.255 \\
Máscara de red               & 255.255.254.0 \\
\bottomrule
\end{tabular}
\end{table}

\begin{table}[H]
\centering
\caption*{Subred 3: 192.168.132.0/23}
\begin{tabular}{@{}ll@{}}
\toprule
ID de red                   & 192.168.132.0 \\ 
Primera dirección utilizable & 192.168.132.1 \\
Última dirección utilizable  & 192.168.133.254 \\
Broadcast                   & 192.168.133.255 \\
Máscara de red               & 255.255.254.0 \\
\bottomrule
\end{tabular}
\end{table}

\begin{table}[H]
\centering
\caption*{Subred 4: 192.168.134.0/23}
\begin{tabular}{@{}ll@{}}
\toprule
ID de red                   & 192.168.134.0 \\ 
Primera dirección utilizable & 192.168.134.1 \\
Última dirección utilizable  & 192.168.135.254 \\
Broadcast                   & 192.168.135.255 \\
Máscara de red               & 255.255.254.0 \\
\bottomrule
\end{tabular}
\end{table}

\begin{table}[H]
\centering
\caption*{Subred 5: 192.168.136.0/23}
\begin{tabular}{@{}ll@{}}
\toprule
ID de red                   & 192.168.136.0 \\ 
Primera dirección utilizable & 192.168.136.1 \\
Última dirección utilizable  & 192.168.137.254 \\
Broadcast                   & 192.168.137.255 \\
Máscara de red               & 255.255.254.0 \\
\bottomrule
\end{tabular}
\end{table}

\begin{table}[H]
\centering
\caption*{Subred 6: 192.168.138.0/23}
\begin{tabular}{@{}ll@{}}
\toprule
ID de red                   & 192.168.138.0 \\ 
Primera dirección utilizable & 192.168.138.1 \\
Última dirección utilizable  & 192.168.139.254 \\
Broadcast                   & 192.168.139.255 \\
Máscara de red               & 255.255.254.0 \\
\bottomrule
\end{tabular}
\end{table}

\begin{table}[H]
\centering
\caption*{Subred 7: 192.168.140.0/23}
\begin{tabular}{@{}ll@{}}
\toprule
ID de red                   & 192.168.140.0 \\ 
Primera dirección utilizable & 192.168.140.1 \\
Última dirección utilizable  & 192.168.141.254 \\
Broadcast                   & 192.168.141.255 \\
Máscara de red               & 255.255.254.0 \\
\bottomrule
\end{tabular}
\end{table}

\subsection{Evidencias procedimientos}
En la Figura \ref{fig:procedimientoSubredes} pueden verse los diagramas utilizados para el cálculo de la 
información de las subredes para este ejercicio.

\begin{figure}[H]
  \begin{center}
    \includegraphics[width=0.8\textwidth]{procedimientoSubredes.jpeg}
  \end{center}
  \caption{procedimiento cálculo de información subredes}\label{fig:procedimientoSubredes}
\end{figure}

\section*{Tabla de direccionamiento}

Todas las subredes utilizan la máscara \textbf{255.255.254.0 (/23)}.\
La primera dirección usable de cada subred se asigna como \textbf{gateway}.\
Los routers no tienen gateway configurado.

\bigskip

\begin{center}
\renewcommand{\arraystretch}{1.2}
\begin{tabular}{|l|l|l|l|l|}
\hline
\textbf{Dispositivo} & \textbf{Interfaz} & \textbf{IP} & \textbf{Máscara de red} & \textbf{Gateway} \\
\hline
\multicolumn{5}{|c|}{\textbf{Router R1}} \\
\hline
R1 & GigabitEthernet0/0 & 192.168.128.1 & 255.255.254.0 & -- \\
R1 & GigabitEthernet0/1 & 192.168.130.1 & 255.255.254.0 & -- \\
R1 & Serial0/0/0        & 192.168.138.1 & 255.255.254.0 & -- \\
R1 & Serial0/0/1        & 192.168.140.1 & 255.255.254.0 & -- \\
\hline
\multicolumn{5}{|c|}{\textbf{Router R2}} \\
\hline
R2 & Serial0/0/0        & 192.168.138.2 & 255.255.254.0 & -- \\
R2 & GigabitEthernet0/0 & 192.168.134.1 & 255.255.254.0 & -- \\
R2 & GigabitEthernet0/1 & 192.168.136.1 & 255.255.254.0 & -- \\
\hline
\multicolumn{5}{|c|}{\textbf{Router R3}} \\
\hline
R3 & Serial0/0/0        & 192.168.140.2 & 255.255.254.0 & -- \\
R3 & GigabitEthernet0/0 & 192.168.132.1 & 255.255.254.0 & -- \\
\hline
\multicolumn{5}{|c|}{\textbf{Switches (IP de administración)}} \\
\hline
S1 & VLAN 1 & 192.168.128.3 & 255.255.254.0 & 192.168.128.1 \\
S2 & VLAN 1 & 192.168.128.2 & 255.255.254.0 & 192.168.128.1 \\
S3 & VLAN 1 & 192.168.134.2 & 255.255.254.0 & 192.168.134.1 \\
S4 & VLAN 1 & 192.168.136.2 & 255.255.254.0 & 192.168.136.1 \\
S5 & VLAN 1 & 192.168.132.2 & 255.255.254.0 & 192.168.132.1 \\
S6 & VLAN 1 & 192.168.130.2 & 255.255.254.0 & 192.168.130.1 \\
\hline
\multicolumn{5}{|c|}{\textbf{PCs}} \\
\hline
PCA & NIC & 192.168.128.10 & 255.255.254.0 & 192.168.128.1 \\
PCB & NIC & 192.168.128.11 & 255.255.254.0 & 192.168.128.1 \\
PCC & NIC & 192.168.128.12 & 255.255.254.0 & 192.168.128.1 \\
PCD & NIC & 192.168.134.10 & 255.255.254.0 & 192.168.134.1 \\
PCE & NIC & 192.168.134.11 & 255.255.254.0 & 192.168.134.1 \\
PCF & NIC & 192.168.136.10 & 255.255.254.0 & 192.168.136.1 \\
PCG & NIC & 192.168.136.11 & 255.255.254.0 & 192.168.136.1 \\
PCH & NIC & 192.168.132.10 & 255.255.254.0 & 192.168.132.1 \\
PCI & NIC & 192.168.132.11 & 255.255.254.0 & 192.168.132.1 \\
\hline
\multicolumn{5}{|c|}{\textbf{Servidores}} \\
\hline
Serv1 & NIC & 192.168.130.10 & 255.255.254.0 & 192.168.130.1 \\
Serv2 & NIC & 192.168.130.11 & 255.255.254.0 & 192.168.130.1 \\
\hline
\end{tabular}
\end{center}

\end{document}
