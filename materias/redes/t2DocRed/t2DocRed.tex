\documentclass[12pt]{article}

\usepackage[utf8]{inputenc}
\usepackage{amsmath, amssymb, amsfonts, cancel}
\usepackage{graphicx}
\usepackage[letterpaper, left=1in, right=1in, top=1in, bottom=1in]{geometry}
\usepackage{booktabs}
\usepackage{multirow}
\usepackage{float}
\usepackage{caption}
\usepackage{colortbl}
\usepackage{xcolor}
\definecolor{paleYellow}{RGB}{255,255,180}
\usepackage{physics}
\usepackage{hyperref}
\usepackage[spanish]{babel}

\title{
Tarea 1: Documentación de red\\[0.5em]
\large Administración de redes
}

\author{
Jorge Alberto Suarez Saldana\\
\small Matrícula: 355992\\
\small Profesor: Raymundo Antonio González Grimaldo
}

\date{\today}

\begin{document}
\maketitle

\section{Topología de la red}
Podemos observar en la Figura \ref{fig:topologia} un diagrama de la topología de la red en el modo lógico del simulador
Cisco Packet Tracer.
\begin{figure}[H]
  \centering
  \includegraphics[width=0.8\textwidth]{diagramaTopologia.png}
  \caption{Topología de la red}\label{fig:topologia}
\end{figure}

\section{Generación de subredes}
\subsection{IPv4}
\end{document}
