\documentclass[12pt]{article}

\usepackage[utf8]{inputenc}
\usepackage{amsmath, amssymb, amsfonts, cancel}
\usepackage{graphicx}
\usepackage[letterpaper, left=1in, right=1in, top=1in, bottom=1in]{geometry}
\usepackage{booktabs}
\usepackage{multirow}
\usepackage{float}
\usepackage{caption}
\usepackage{colortbl}
\usepackage{xcolor}
\definecolor{paleYellow}{RGB}{255,255,180}
\usepackage{physics}
\usepackage{hyperref}
\usepackage[spanish]{babel}

\title{
Tarea 1: Documentación de red IPv6\\[0.5em]
\large Administración de redes
}

\author{
Jorge Alberto Suarez Saldana\\
\small Matrícula: 355992\\
\small Profesor: Raymundo Antonio González Grimaldo
}

\date{\today}

\begin{document}
\maketitle

\section{Topología de la red}
Podemos observar en la Figura \ref{fig:topologia} un diagrama de la topología de la red en el modo lógico del simulador
Cisco Packet Tracer.
\begin{figure}[H]
  \centering
  \includegraphics[width=0.8\textwidth]{topologia.png}
  \caption{Topología de la red}\label{fig:topologia}
\end{figure}

\section{Generación de subredes}
\subsection{IPv6}
IPv6 es la versión más reciente del protocolo IP, diseñada para reemplazar IPv4 y solucionar principalmente la 
escasez de direcciones IP.

IPv6 utiliza direcciones de 128 bits, organizados en 8 \textit{hextetos} 
(4 digitos hexadecimales) separados por dos puntos.

\section{Prefijo de red}
Funciona igual y con la misma notación de diagonal de IPv4, indicando con una diagonal el número de bits que 
perteneces a la parte de red.

En la práctica, es recomendado siempre utilizar el prefijo /64 al trabajar con redes en IPv6, principalmente por facilidad, 
pues el numero de host disponibles es enorme.

\subsection{Subneteo}
En el ejercicio nos piden dividir la red principal en 6 subredes (delimitadas por los routers).
Para hacer esto, puesto que el prefijo de red es /64 (4to hexteto), podemos usar este último como \textbf{identificador de subred}.
Al ser hexadecimal, lon numeros avanzan naturalmente de 0 a F.
Obtenemos entonces las siguientes subredes:
\begin{enumerate}
  \item 2001:AF:0:0::/64
  \item 2001:AF:0:1::/64
  \item 2001:AF:0:2::/64
  \item 2001:AF:0:3::/64
  \item 2001:AF:0:4::/64
  \item 2001:AF:0:5::/64
\end{enumerate}

En IPv6 no existen las direcciones de broadcast ni el concepto formal de primera y última dirección utilizable.
El direccionamiento se basa únicamente en prefijos, siendo común el uso de /64, donde los primeros 64 bits identifican la red 
y el resto los hosts.

\end{document}
