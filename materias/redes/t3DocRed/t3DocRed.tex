\documentclass[12pt]{article}

\usepackage[utf8]{inputenc}
\usepackage{amsmath, amssymb, amsfonts, cancel}
\usepackage{graphicx}
\usepackage[letterpaper, left=1in, right=1in, top=1in, bottom=1in]{geometry}
\usepackage{booktabs}
\usepackage{multirow}
\usepackage{float}
\usepackage{caption}
\usepackage{colortbl}
\usepackage{xcolor}
\definecolor{paleYellow}{RGB}{255,255,180}
\usepackage{physics}
\usepackage{hyperref}
\usepackage[spanish]{babel}

\title{
Tarea 3: Configuración de VLANs\\[0.5em]
\large Administración de redes
}

\author{
Jorge Alberto Suarez Saldana\\
\small Matrícula: 355992\\
\small Profesor: Raymundo Antonio González Grimaldo
}

\date{\today}

\begin{document}
\maketitle

\section{Topología de la red}
Podemos observar en la Figura \ref{fig:topologia} un diagrama de la topología de la red en el modo lógico del simulador
Cisco Packet Tracer.
\begin{figure}[H]
  \centering
  \includegraphics[width=0.8\textwidth]{diagramaTopologia.png}
  \caption{Topología de la red}\label{fig:topologia}
\end{figure}

\section{Tabla de direccionamiento}

\bigskip

\begin{center}
\renewcommand{\arraystretch}{1.2}
\begin{tabular}{|l|l|l|l|l|}
\hline
\textbf{Dispositivo} & \textbf{VLAN} & \textbf{IP} & \textbf{Máscara de red} & \textbf{Gateway} \\
\hline
\multicolumn{5}{|c|}{\textbf{VLAN 10 - Alumnos}} \\
\hline
R1 & 10 & 52.10.10.1 & 255.255.255.0 & -- \\
PC-A & 10 & 52.10.10.10 & 255.255.255.0 & 52.10.10.1 \\
PC-C & 10 & 52.10.10.11 & 255.255.255.0 & 52.10.10.1 \\
\hline
\multicolumn{5}{|c|}{\textbf{VLAN 20 - Profesores}} \\
\hline
R1 & 20 & 52.10.20.1 & 255.255.255.0 & -- \\
PC-B & 20 & 52.10.20.10 & 255.255.255.255.0 & 52.10.20.1 \\
PC-D & 20 & 52.10.20.11 & 255.255.255.0 & 52.10.20.1 \\
\hline
\multicolumn{5}{|c|}{\textbf{VLAN 40 - Admin (Nativa)}} \\
\hline
R1 & 40 & 52.10.40.1 & 255.255.255.0 & -- \\
PC-E & 40 & 52.10.40.10 & 255.255.255.0 & 52.10.40.1 \\
PC-F & 40 & 52.10.40.11 & 255.255.255.0 & 52.10.40.1 \\
\hline
\multicolumn{5}{|c|}{\textbf{VLAN 50 - RH}} \\
\hline
R1 & 50 & 192.168.50.1 & 255.255.255.0 & -- \\
PC-G & 50 & 192.168.50.10 & 255.255.255.0 & 192.168.50.1\\
PC-H & 50 & 192.168.50.11 & 255.255.255.0 & 192.168.50.1\\
\hline
\multicolumn{5}{|c|}{\textbf{VLAN 70 - Admin (Nativa)}} \\
\hline
R1 & 70 & 192.168.70.1 & 255.255.255.0 & -- \\
PC-I & 70 & 192.168.70.10 & 255.255.255.0 & 192.168.70.1\\
PC-J & 70 & 192.168.70.11 & 255.255.255.0 & 192.168.70.1\\
\hline
\end{tabular}
\end{center}

\section{Configuración del Router}
EN la figura \ref{fig:ipInterfaceBriefR1} podemos observar la ejecución del comando \textit{show ip interface brief} en el router R1, en la que se nota la configuración 
de las interfaces y sus respectivas VLAN aspi como su estado.
\begin{figure}[H]
  \centering
  \includegraphics[width=0.8\textwidth]{ipInterfaceBriefR1.png}
  \caption{Ejecución de comando show ip interface brief en router R1}\label{fig:ipInterfaceBriefR1}
\end{figure}

\section{Configuración de los Switchers}
En las figuras \ref{fig:showVlanBriefS1}, \ref{fig:showVlanBriefS2}, \ref{fig:showVlanBriefS3} y \ref{fig:showVlanBriefS4} podemos observar la ejecución del 
comando \textit{show vlan brief} en los switch 1 - 4 respectivamente, en las que se nota la configuración de las VLANs, su estado y el puerto físico al 
que están asignadas.

\begin{figure}[H]
  \centering
  \includegraphics[width=0.8\textwidth]{showVlanBriefS1.png}
  \caption{Ejecución del comando show vlan brief en el switch S1}\label{fig:showVlanBriefS1}
\end{figure}

\begin{figure}[H]
  \centering
  \includegraphics[width=0.8\textwidth]{showVlanBriefS2.png}
  \caption{Ejecución del comando show vlan brief en el switch S2}\label{fig:showVlanBriefS2}
\end{figure}

\begin{figure}[H]
  \centering
  \includegraphics[width=0.8\textwidth]{showVlanBriefS3.png}
  \caption{Ejecución del comando show vlan brief en el switch S3}\label{fig:showVlanBriefS3}
\end{figure}

\begin{figure}[H]
  \centering
  \includegraphics[width=0.8\textwidth]{showVlanBriefS4.png}
  \caption{Ejecución del comando show vlan brief en el switch S4}\label{fig:showVlanBriefS4}
\end{figure}

\section{Pruebas de conectividad}
En la figura \ref{fig:pruebasConect} se pueden observar el registro de las 6 pruebas de conectividad realizadas entre diferetes dispositivos de la red, específicados 
entre las columnas 3 y 4.

\begin{figure}[H]
  \centering
  \includegraphics[width=0.8\textwidth]{pruebasConect.png}
  \caption{Pruebas de conectividad entre diferentes dispositivos}\label{fig:pruebasConect}
\end{figure}

\section{Conclusión}
Las VLAN permiten separar de forma lógica equipos dentro de una misma red física para simplificar la gestión de equipos con necesidades conjuntas.
Mientras hacía la práctica, tuve un problema en el que, mientras las computadoras pertenecientes a la VLAN 40 (Admin) estaban bien configuradas, en sus 
correspondientes interfaces en el switch configuré una VLAN diferente, por lo que todas las pruebas de conectividad que involucraran dichos host fallaban.

Al consultar con el profesor, este realizó un debbugging interesante en el que encendió una configuración que mostraba a qué interfaz física estaba conectado cada 
dispositivo de la red, y después comparo esa información con la configuración de vlans del switch, dándose cuenta de esta forma. Me llevo esta técnica para aplicarla después, 
procurando por supuesto tener cuidado con la configuración inicial de la red.

Finalmente, un descubrimiento para mí en la interfaz del simulador Cisco Packet Tracer es la de agilizar los \textit{ping}s entre dispositivos 
utilizando el ícono de sobre para carta, lo que también permite visualizar en el modo simulador el viaje de dicho paquete de información.
Debe decirse que este método puede regresar un \textit{failed} las primeras veces que se ejecuta, por lo que es recomendable probar cada conección al menos 5 veces.

\end{document}
