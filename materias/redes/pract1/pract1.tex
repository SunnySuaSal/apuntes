\documentclass[12pt]{article}

\usepackage[utf8]{inputenc}
\usepackage{amsmath, amssymb, amsfonts, cancel}
\usepackage{graphicx}
\usepackage[letterpaper, left=1in, right=1in, top=1in, bottom=1in]{geometry}
\usepackage{booktabs}
\usepackage{multirow}
\usepackage{float}
\usepackage{caption}
\usepackage{colortbl}
\usepackage{xcolor}
\definecolor{paleYellow}{RGB}{255,255,180}
\usepackage{physics}
\usepackage{hyperref}
\usepackage[spanish]{babel}

\title{
Práctica 1: Evaluación Integral de Diseño y Optimización de Redes con VLSM\\[0.5em]
\large Administración de redes
}

\author{
Jorge Alberto Suarez Saldana\\
\small Matrícula: 355992\\
\small Profesor: Raymundo Antonio González Grimaldo
}

\date{\today}

\begin{document}
\maketitle

\section{Generación de subredes}
\begin{table}[H]
\centering
\caption*{Subred 1: Departamento de desarrollo}
\begin{tabular}{@{}ll@{}}
\toprule
ID de red                   & 192.168.8.0 \\ 
Primera dirección utilizable & 192.168.8.1 \\
Última dirección utilizable  & 192.168.9.254 \\
Broadcast                   & 192.168.9.255 \\
Máscara de red               & 255.255.254.0 \\
\bottomrule
\end{tabular}
\end{table}


\begin{table}[H]
\centering
\caption*{Subred 2: Departamento de ventas}
\begin{tabular}{@{}ll@{}}
\toprule
ID de red                   & 192.168.10.0 \\ 
Primera dirección utilizable & 192.168.10.1 \\
Última dirección utilizable  & 192.168.10.126 \\
Broadcast                   & 192.168.10.127 \\
Máscara de red               & 255.255.255.128 \\
\bottomrule
\end{tabular}
\end{table}

\begin{table}[H]
\centering
\caption*{Subred 3: Departamento de soporte}
\begin{tabular}{@{}ll@{}}
\toprule
ID de red                   & 192.168.10.128 \\ 
Primera dirección utilizable & 192.168.10.129 \\
Última dirección utilizable  & 192.168.10.158 \\
Broadcast                   & 192.168.10.159 \\
Máscara de red               & 255.255.255.224 \\
\bottomrule
\end{tabular}
\end{table}

\begin{table}[H]
\centering
\caption*{Subred 4: Departamento de derechos humanos}
\begin{tabular}{@{}ll@{}}
\toprule
ID de red                   & 192.168.10.160 \\ 
Primera dirección utilizable & 192.168.10.161 \\
Última dirección utilizable  & 192.168.10.174 \\
Broadcast                   & 192.168.10.175 \\
Máscara de red               & 255.255.255.240 \\
\bottomrule
\end{tabular}
\end{table}

\begin{table}[H]
\centering
\caption*{Subred 5: Administrador}
\begin{tabular}{@{}ll@{}}
\toprule
ID de red                   & 172.16.10.0 \\ 
Primera dirección utilizable & 172.16.10.1 \\
Última dirección utilizable  & 172.16.10.2 \\
Broadcast                   & 172.16.10.3 \\
Máscara de red               & 255.255.255.252 \\
\bottomrule
\end{tabular}
\end{table}

\begin{center}
\renewcommand{\arraystretch}{1.2}
\begin{tabular}{|l|l|l|l|l|}
\hline
\textbf{Dispositivo} & \textbf{Interfaz} & \textbf{IP} & \textbf{Máscara de red} & \textbf{Gateway} \\
\hline
\multicolumn{5}{|c|}{\textbf{Router R1}} \\
\hline
R1 & Giga 0/0/0 & 192.168.10.161 & 255.255.255.240 & -- \\
R1 & Giga 0/0/1 & 172.16.10.1 & 255.255.255.252 & -- \\
\hline
\multicolumn{5}{|c|}{\textbf{Switch S1}} \\
\hline
S1 & VLAN 1 & 192.168.10.162 & 255.255.255.240 & 192.168.10.161 \\
\hline
\multicolumn{5}{|c|}{\textbf{PCs}} \\
\hline
PC-A & NIC & 192.168.10.174 & 255.255.255.240 & 192.168.10.161 \\
PC-B & NIC & 192.168.10.173 & 255.255.255.240 & 192.168.10.161 \\
PC-C & NIC & 192.168.10.172 & 255.255.255.240 & 192.168.10.161 \\
PC-Admin & NIC & 172.16.10.2 & 255.255.255.252 & 172.16.10.1 \\
\hline
\end{tabular}
\end{center}


\section{Pruebas de conectividad de los equipos}
Tras configurar correctamente mediante la CLI los equipos de la red, podemos realizar las siguientes pruebas de conectividad:

\begin{figure}[H]
  \begin{center}
    \includegraphics[width=0.8\textwidth]{ping_PCA-PCB.png}
  \end{center}
  \caption{Ping de la PC-A a la PC-B}\label{fig:ping_PCA-PCB}
\end{figure}

\begin{figure}[H]
  \begin{center}
    \includegraphics[width=0.8\textwidth]{ping_PCB-R1.png}
  \end{center}
  \caption{Ping de la PC-B al R1}\label{fig:ping_PCB-R1}
\end{figure}

\begin{figure}[H]
  \begin{center}
    \includegraphics[width=0.8\textwidth]{ping_PCADM-PCC.png}
  \end{center}
  \caption{Ping de la PC-ADMIN a la PC-C}\label{fig:ping_PCADM-PCC}
\end{figure}

\section{Salida del comando show running-config de cada dispositivo}
Para confirmar que los equipos de red estén correctamente configurados, podemos ejecutar comandos como \textit{show running-config} 
y \textit{show ip interface brief}.

\begin{figure}[H]
  \begin{center}
    \includegraphics[width=0.8\textwidth]{runningConfig_S1.png}
  \end{center}
  \caption{Comando show running-config en S1}\label{fig:runningConfig_S1}
\end{figure}

\begin{figure}[H]
  \begin{center}
    \includegraphics[width=0.8\textwidth]{interface_S1.png}
  \end{center}
  \caption{Comando show ip interface brief en S1}\label{fig:interface_S1}
\end{figure}

\begin{figure}[H]
  \begin{center}
    \includegraphics[width=0.8\textwidth]{runningConfig_R1.png}
  \end{center}
  \caption{Comando show running-config en R1}\label{fig:runningConfig_R1}
\end{figure}

\begin{figure}[H]
  \begin{center}
    \includegraphics[width=0.8\textwidth]{interface_R1.png}
  \end{center}
  \caption{Comando show ip interface brief en R1}\label{fig:interface_R1}
\end{figure}

\section{Equipo físico}
La práctica, además de en el simulador de Cisco Packet Tracer, también se realizó en equipo físico en el laboratorio de redes 
y fue evaluada por el profesor.

A continuación, las imágenes de evidencia de la realización de dicha práctica en equipo físico:

\begin{figure}[H]
  \begin{center}
    \includegraphics[width=0.8\textwidth]{evidencia1.JPG}
  \end{center}
  \caption{Prueba de conección 1}\label{fig:evidencia1}
\end{figure}

\begin{figure}[H]
  \begin{center}
    \includegraphics[width=0.8\textwidth]{evidencia2.JPG}
  \end{center}
  \caption{Prueba de conección 2}\label{fig:evidencia2}
\end{figure}

\begin{figure}[H]
  \begin{center}
    \includegraphics[width=0.8\textwidth]{evidencia3.JPG}
  \end{center}
  \caption{Prueba de conección 3}\label{fig:evidencia3}
\end{figure}

\begin{figure}[H]
  \begin{center}
    \includegraphics[width=0.8\textwidth]{evidencia4.JPG}
  \end{center}
  \caption{Prueba de conección 4}\label{fig:evidencia4}
\end{figure}

\begin{figure}[H]
  \begin{center}
    \includegraphics[width=0.8\textwidth]{evidencia5.JPG}
  \end{center}
  \caption{Prueba de conección 5}\label{fig:evidencia5}
\end{figure}

\begin{figure}[H]
  \begin{center}
    \includegraphics[width=0.8\textwidth]{evidencia6.jpg}
  \end{center}
  \caption{Prueba de equipo físico en el que trabajamos}\label{fig:evidencia6}
\end{figure}


\end{document}

