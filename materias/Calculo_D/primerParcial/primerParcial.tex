
\documentclass[12pt]{article} %Tipo de documento (puede ser book, report, etc.)

\usepackage[utf8]{inputenc} %Codificación de caracteres (UTF-8)
\usepackage{amsmath, amssymb, physics} %Paquetes para expresiones matemáticas
\usepackage{graphicx} %Para insertar imágenes
\usepackage[spanish]{babel} %Optimiza el typesetting para documentos en español
\usepackage[letterpaper, left=1in, right=1in, top=1in, bottom=1in]{geometry}
%Para abarcar más hoja horizontalmente
\usepackage{booktabs} %Optimiza trabajar con tablas y agrega algunos comandos
\usepackage{multirow} %Para crear celdas tabulares que abarcan múltiples filas
\usepackage{float} %Mayor control sobre dónde se colocan las figuras y tablas
\usepackage{caption} %Mayor control de las captions de figuras y tablas
\usepackage{colortbl} %Para colores en tablas
\usepackage{xcolor} %Opcional, pero útil para definir colores personalizados
\definecolor{paleYellow}{RGB}{255, 255, 180} %Ejemplo con amarillo pálido
\usepackage{hyperref}
\usepackage{enumitem}

%Preámbulo
\title{Apuntes Cálculo D Primer parcial}
\author{Suárez Saldaña, Jorge Alberto \\ Matrícula: 355992}
\date{\today}

\begin{document}
\maketitle

\section{Definiciones}

\begin{description}
  \item[Ecuación Diferencial:] Es aquella ecuación que contiene derivadas o diferenciales.

  \item[Orden de una derivada:] Se refiere a cuántas veces se ha derivado una función.
    \begin{table}[H]
      \centering
    \begin{tabular}{cc}
      $\dv{y}{x} \rightarrow \text{1er orden}$ & $\dv[3]{y}{x} \rightarrow \text{3er orden}$ \\[0.5em]
      $\dv[2]{y}{x} \rightarrow \text{2do orden}$ & $\dv[n]{y}{x} \rightarrow \text{n orden}$
    \end{tabular}
  \end{table}

  \item[Orden de una ecuación diferencial:] Es el orden de la derivada de mayor orden contenida en la ecuación.
    \begin{table}[H]
      \centering
    \begin{tabular}{cc}
      $F(x,y,y^{\prime}) = 0 \rightarrow \text{Primer orden}$ & 
      $F(x,y,y^{\prime},y^{\prime \prime}) = 0 \rightarrow \text{Segundo orden}$ \\[0.5em]
      $F(x,y,y^{\prime},y^{\prime \prime},y^{\prime \prime \prime}) = 0 \rightarrow \text{Tercer orden}$
    \end{tabular}
  \end{table}

\item[Grado de una ecuación diferencial:] Es la potencia a la que está elevada la derivada de mayor orden en la ecuación, 
  siempre y cuando esta esté dada en forma polinomial.

  Diferenciándose en 2 grupos principales:
  \begin{itemize}
    \item Lineales: 
      \begin{itemize}
        \item La variable dependiente $y$ y todas sus derivadas son de 1er grado.
        \item Cada coeficiente de $y$ y sus derivadas dependen solamente de la variable independiente x.
      \end{itemize}
    \item No lineales: Las que no cumplen con las propiedades anteriores.
  \end{itemize}
\end{description}

También podemos distinguir entre dos \textbf{tipos} principales de ecuaciones diferenciales:
\begin{description}
  \item[Ordinarias:] La ecuación diferencial contiene derivadas de una o más variables dependientes 
    \underline{con respecto a una sola variable independiente}.
  \item[Parciales:] La ecuación diferencial contiene derivadas parciales de una o más variables dependientes
    \underline{con respecto a dos o más variables independientes}.
\end{description}

Obsérvese la tabla \ref{tab:EjemploClasificacion} un ejemplo de clasificación de diferentes ecuaciones diferenciales de acuerdo a las definiciones 
recién estudiadas.

\begin{table}
  \centering
  \caption{Ejemplo de clasificación de ecuaciones diferenciales}
  \label{tab:EjemploClasificacion}
  \begin{tabular}{lccc}
    Ecuación diferencial & Tipo & Orden & Lineal \\[0.5em]
    $\dv{y}{x} = 2e^{-x}$ & Ordinaria & 1 & Sí \\[0.6em]
    $\pdv{y}{t} = \pdv{x}{t} + ky - \pdv{y}{s}$ & Parcial & 1 & Sí* \\[0.6em]
    $x^2y^{\prime \prime} + xy^{\prime} + y = 0$ & Ordinaria & 2 & Sí \\[0.6em]
    $yy^{\prime \prime} + x^2y = x$ & Ordinaria & 2 & No* \\[0.6em]
    $\pdv{y}{t} + \pdv[2]{y}{s} = c$ &Parcial & 2 & Sí* \\[0.6em]
    $x^2 \pdv[2]{y}{x} + x \dv{y}{x} + (x^2 + 9)y = 0$ & Ordinaria & 2 & Sí \\[0.6em]
    $\pdv[4]{v}{t} = kv(\pdv[2]{m}{n})^2$ & Parcial & 4 & No* \\[0.6em]
    $(y^V)^3 - y^{\prime \prime \prime} + y^{\prime \prime} - y^2 = 0$ & Ordinaria & 5 & No \\[0.6em]
    $y^{\prime} + y = \frac{x}{y}$ & Ordinaria & 1 & No \\[0.6em]
    $\sin (y^{\prime}) + y = 0$ & Ordinaria & 1 & No*
  \end{tabular}
\end{table}

Lo que realmente importa para la linealidad es cómo aparecen $y$ (la variable dependiente) y sus derivadas.
No importa: 
\begin{itemize}
  \item Que deriven a $y$ respecto de \textbf{diferentes variables independientes}.
  \item Que haya varias derivadas parciales de $y$ en la misma ecuación.
\end{itemize}
Lo que sí importa:
\begin{enumerate}
\item El grado: Todas las derivadas de $y$ y la propia $y$ deben aparecer en \textbf{primer grado} (sin estar elevadas a potencias diferentes de 1, ni estar multiplicadas entre sí).
\item El tipo de funciones: $y$ y sus derivadas \textbf{no deben estar dentro de funciones no lineales} como $\sin$ , $\cos$ , e, $\ln$ , etc.
\item Los coeficintes: Los coeficientes que acompañan a $y$ o a sus derivadas \textbf{sólo pueden depender de las variables independientes} (en este caso: $x$, $t$, $s$, etc.)
Si el coeficiente depende de $y$ o de una derivada de $y$, entonces la ecuación ya no es lineal.

Una vez aclaradas las propiedades de la linealidad con los ejemplos, podemos pasar al ejercicio de la tabla \ref{tab:EjercicioClasificacion}, que corresponde con la tarea 1.
\end{enumerate}

\begin{table}
  \centering
  \caption{Tarea1: Clasificación de ecuaciones diferenciales}
  \label{tab:EjercicioClasificacion}
  \begin{tabular}{lccc}
    Ecuación diferencial & Tipo & Orden & Lineal \\[0.5em]
    $y^{\prime \prime} + xyy^{\prime} = \sin (x)$ & Ordinaria & 2 & No \\[0.6em]
    $3 \pdv[5]{x}{t} + \pdv[2]{y}{r} = 5$ & Parcial & 5 & Sí \\[0.6em]
    $x^3yy^{\prime \prime \prime} - x^2yy^{\prime \prime} + y = 0$ & Ordinaria & 3 & No \\[0.6em]
    $y^{\prime \prime} + 2x^3y^{\prime} - (x-1)y = x^{\frac{3}{2}}$ & Ordinaria & 2 & Sí \\[0.6em]
    $(\pdv{u}{x})^2 + \pdv[2]{u}{y} = \frac{x}{y}$ &Parcial & 2 & No \\[0.6em]
    $\dv[2]{y}{x} - 2 \dv{y}{x} + y = 0$ & Ordinaria & 2 & Sí \\[0.6em]
    $(1+y)y^{\prime} + 2y = e^x$ & Ordinaria & 1 & No \\[0.6em]
    $\pdv[2]{y}{x} = \pdv[2]{y}{t}$ & Parcial & 2 & Sí \\[0.6em]
    $y^{\prime \prime} + 9y = \sin (y)$ & Ordinaria & 2 & No \\[0.6em]
    $\dv{y}{x} = \sqrt{1 - (\pdv[2]{y}{x})^2}$ & Ordinaria & 2 & No
  \end{tabular}
\end{table}

\end{document}
