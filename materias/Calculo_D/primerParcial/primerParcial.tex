
\documentclass[12pt]{article} %Tipo de documento (puede ser book, report, etc.)

\usepackage[utf8]{inputenc} %Codificación de caracteres (UTF-8)
\usepackage{amsmath, amssymb, physics} %Paquetes para expresiones matemáticas
\usepackage{graphicx} %Para insertar imágenes
\usepackage[spanish]{babel} %Optimiza el typesetting para documentos en español
\usepackage[letterpaper, left=1in, right=1in, top=1in, bottom=1in]{geometry}
%Para abarcar más hoja horizontalmente
\usepackage{booktabs} %Optimiza trabajar con tablas y agrega algunos comandos
\usepackage{multirow} %Para crear celdas tabulares que abarcan múltiples filas
\usepackage{float} %Mayor control sobre dónde se colocan las figuras y tablas
\usepackage{caption} %Mayor control de las captions de figuras y tablas
\usepackage{colortbl} %Para colores en tablas
\usepackage{xcolor} %Opcional, pero útil para definir colores personalizados
\usepackage{cancel} %Para cencelar elementos en ecuaciones
\definecolor{paleYellow}{RGB}{255, 255, 180} %Ejemplo con amarillo pálido
\usepackage{hyperref}
\usepackage{enumitem}
\usepackage{amsthm} %Para definir entornos personalizados
\usepackage{cleveref} %Agrega el nombre del environment a la referencia
\usepackage{xparse} %Para facilitar la definicion de comandos

%Definir entornos personalizados para tareas y ejercicios
\theoremstyle{remark} %Para que el estilo de fuente siga normal
\newtheorem{tarea}{Tarea}[section] %Numerado por seccion
%Enumerador personalizado para los ejercicios dentro de las tareas
\newcounter{ejercicio}[tarea]
\renewcommand{\theejercicio}{\thetarea.\arabic{ejercicio}}

\NewDocumentEnvironment{ejercicio}{o}{
  \refstepcounter{ejercicio}
  \noindent\textbf{Ejercicio~
    \IfValueTF{#1}{#1}{\theejercicio}
  }
  \par
}{}
%Configuramos referencias
\crefname{tarea}{Tarea}{Tareas}
\crefname{ejercicio}{Ejercicio}{Ejercicios}

%Environment para ejemplos
\newtheorem{ejemplo}{Ejemplo}[section]

%Definir comandos para abreviar
\newcommand{\ed}{ecuación diferencial}

%Preámbulo
\title{Apuntes Cálculo D Primer parcial}
\author{Suárez Saldaña, Jorge Alberto \\ Matrícula: 355992}
\date{\today}

\begin{document}
\maketitle

\section{Definiciones}

\begin{description}
  \item[Ecuación Diferencial:] Es aquella ecuación que contiene derivadas o diferenciales.

  \item[Orden de una derivada:] Se refiere a cuántas veces se ha derivado una función.
    \begin{table}[H]
      \centering
    \begin{tabular}{cc}
      $\dv{y}{x} \rightarrow \text{1er orden}$ & $\dv[3]{y}{x} \rightarrow \text{3er orden}$ \\[0.5em]
      $\dv[2]{y}{x} \rightarrow \text{2do orden}$ & $\dv[n]{y}{x} \rightarrow \text{n orden}$
    \end{tabular}
  \end{table}

  \item[Orden de una ecuación diferencial:] Es el orden de la derivada de mayor orden contenida en la ecuación.
    \begin{table}[H]
      \centering
    \begin{tabular}{cc}
      $F(x,y,y^{\prime}) = 0 \rightarrow \text{Primer orden}$ & 
      $F(x,y,y^{\prime},y^{\prime \prime}) = 0 \rightarrow \text{Segundo orden}$ \\[0.5em]
      $F(x,y,y^{\prime},y^{\prime \prime},y^{\prime \prime \prime}) = 0 \rightarrow \text{Tercer orden}$
    \end{tabular}
  \end{table}

\item[Grado de una ecuación diferencial:] Es la potencia a la que está elevada la derivada de mayor orden en la ecuación, 
  siempre y cuando esta esté dada en forma polinomial.

  Diferenciándose en 2 grupos principales:
  \begin{itemize}
    \item Lineales: 
      \begin{itemize}
        \item La variable dependiente $y$ y todas sus derivadas son de 1er grado.
        \item Cada coeficiente de $y$ y sus derivadas dependen solamente de la variable independiente x.
      \end{itemize}
    \item No lineales: Las que no cumplen con las propiedades anteriores.
  \end{itemize}
\end{description}

También podemos distinguir entre dos \textbf{tipos} principales de ecuaciones diferenciales:
\begin{description}
  \item[Ordinarias:] La ecuación diferencial contiene derivadas de una o más variables dependientes 
    \underline{con respecto a una sola variable independiente}.
  \item[Parciales:] La ecuación diferencial contiene derivadas parciales de una o más variables dependientes
    \underline{con respecto a dos o más variables independientes}.
\end{description}

Obsérvese la tabla \ref{tab:EjemploClasificacion} un ejemplo de clasificación de diferentes ecuaciones diferenciales de acuerdo a las definiciones 
recién estudiadas.

\begin{ejemplo}
\begin{table}
  \centering
  \caption{Ejemplo de clasificación de ecuaciones diferenciales}
  \label{tab:EjemploClasificacion}
  \begin{tabular}{lccc}
    Ecuación diferencial & Tipo & Orden & Lineal \\[0.5em]
    $\dv{y}{x} = 2e^{-x}$ & Ordinaria & 1 & Sí \\[0.6em]
    $\pdv{y}{t} = \pdv{x}{t} + ky - \pdv{y}{s}$ & Parcial & 1 & Sí \\[0.6em]
    $x^2y^{\prime \prime} + xy^{\prime} + y = 0$ & Ordinaria & 2 & Sí \\[0.6em]
    $yy^{\prime \prime} + x^2y = x$ & Ordinaria & 2 & No \\[0.6em]
    $\pdv{y}{t} + \pdv[2]{y}{s} = c$ &Parcial & 2 & Sí \\[0.6em]
    $x^2 \pdv[2]{y}{x} + x \dv{y}{x} + (x^2 + 9)y = 0$ & Ordinaria & 2 & Sí \\[0.6em]
    $\pdv[4]{v}{t} = kv(\pdv[2]{m}{n})^2$ & Parcial & 4 & No \\[0.6em]
    $(y^V)^3 - y^{\prime \prime \prime} + y^{\prime \prime} - y^2 = 0$ & Ordinaria & 5 & No \\[0.6em]
    $y^{\prime} + y = \frac{x}{y}$ & Ordinaria & 1 & No \\[0.6em]
    $\sin (y^{\prime}) + y = 0$ & Ordinaria & 1 & No
  \end{tabular}
\end{table}
\end{ejemplo}

Lo que realmente importa para la linealidad es cómo aparecen $y$ (la variable dependiente) y sus derivadas.
No importa: 
\begin{itemize}
  \item Que deriven a $y$ respecto de \textbf{diferentes variables independientes}.
  \item Que haya varias derivadas parciales de $y$ en la misma ecuación.
\end{itemize}
Lo que sí importa:
\begin{enumerate}
\item El grado: Todas las derivadas de $y$ y la propia $y$ deben aparecer en \textbf{primer grado} (sin estar elevadas a potencias diferentes de 1, ni estar multiplicadas entre sí).
\item El tipo de funciones: $y$ y sus derivadas \textbf{no deben estar dentro de funciones no lineales} como $\sin$ , $\cos$ , e, $\ln$ , etc.
\item Los coeficientes: Los coeficientes que acompañan a $y$ o a sus derivadas \textbf{sólo pueden depender de las variables independientes} (en este caso: $x$, $t$, $s$, etc.)
\item Si el coeficiente depende de $y$ o de una derivada de $y$, entonces la ecuación ya no es lineal.
\end{enumerate}

Procedamos entonces a realizar los ejercicios correspondientes a la tarea \cref{tarea:clasificacion}

\begin{tarea} \label{tarea:clasificacion}
\begin{table}[H]
  \centering
  \caption{Tarea1: Clasificación de ecuaciones diferenciales}
  \label{tab:EjercicioClasificacion}
  \begin{tabular}{lccc}
    Ecuación diferencial & Tipo & Orden & Lineal \\[0.5em]
    $y^{\prime \prime} + xyy^{\prime} = \sin (x)$ & Ordinaria & 2 & No \\[0.6em]
    $3 \pdv[5]{x}{t} + \pdv[2]{y}{r} = 5$ & Parcial & 5 & Sí \\[0.6em]
    $x^3yy^{\prime \prime \prime} - x^2yy^{\prime \prime} + y = 0$ & Ordinaria & 3 & No \\[0.6em]
    $y^{\prime \prime} + 2x^3y^{\prime} - (x-1)y = x^{\frac{3}{2}}$ & Ordinaria & 2 & Sí \\[0.6em]
    $(\pdv{u}{x})^2 + \pdv[2]{u}{y} = \frac{x}{y}$ &Parcial & 2 & No \\[0.6em]
    $\dv[2]{y}{x} - 2 \dv{y}{x} + y = 0$ & Ordinaria & 2 & Sí \\[0.6em]
    $(1+y)y^{\prime} + 2y = e^x$ & Ordinaria & 1 & No \\[0.6em]
    $\pdv[2]{y}{x} = \pdv[2]{y}{t}$ & Parcial & 2 & Sí \\[0.6em]
    $y^{\prime \prime} + 9y = \sin (y)$ & Ordinaria & 2 & No \\[0.6em]
    $\dv{y}{x} = \sqrt{1 - (\pdv[2]{y}{x})^2}$ & Ordinaria & 2 & No
  \end{tabular}
\end{table}
\end{tarea}

\section{Tipos de soluciones de una ecuación diferencial}
La \textbf{solución} de una ecuación diferencial es una función que \textbf{no contiene derivadas} y 
que satisface a dicha ecuación; es decir, al sustituir la función y sus derivadas en la ecuación diferencial 
resulta una \textbf{identidad.}

Una \textbf{solución general} para una ecuación diferencial es la función que satisface a la ecuación y que contiene una o más 
\textbf{constantes arbitrarias} (obtenidas de las sucesivas integraciones).

Una \textbf{solución particular} para una ecuación diferencial es la función que satisface a la ecuación y 
cuyas constantes arbitrarias toman un \textbf{valor específico}.

\begin{ejemplo}
  La función $x + y^2 =c$ es la solución general de la ecuación diferencial \[\dv{y}{x} = - \frac{1}{2y}\]
  Derivando implícitamente $x + y^2 = c$ obtenemos $1+2yy^{\prime} = 0$.
  Despejando $y^{\prime}$ obtenemos \[y^{\prime} = \dv{y}{x} = - \frac{1}{2y}\]
  Al sustituir en la ecuación diferencial obtenemos \[- \frac{1}{2y} = - \frac{1}{2y}\]
  $\therefore$ Se obtiene una identidad.
\end{ejemplo}

\begin{ejemplo}
  La función $y = e^{-x} + 8$ es la solución particular de la \ed $y^{\prime} + e^{-x} = 0$.
  Derivando la solución y despejando $y^{\prime}$ se obtiene \[y^{\prime} = -e^{-x}\]
  Sustituyendo en la \ed : \[-e^{-x} + e^{-x} = 0\]
  $\therefore$ Se obtiene la identidad $0=0$.
\end{ejemplo}

\begin{ejemplo}
  La función $y = 3x^2 + C_1x + C_2$ es la solución general de la \ed \\ $y^{\prime \prime} = 6$ porque 
  \begin{align*}
    y^{\prime} &= 3(2x) + C_1 (1) + 0 \\
    y^{\prime} &= 6x + C_1 \\
    y^{\prime \prime} &= 6(1) + 0 \\
    y^{\prime \prime} &= 6
  \end{align*}
  Sustituyendo en la ecuación diferencial: $\therefore 6 = 6$
\end{ejemplo}

\begin{ejemplo}
  La función $y = e^x (3 \cos (2x) + \sin (2x))$ es la solución particular \\
  de la \ed $y^{\prime \prime} - 2y^{\prime} + 5y = 0$, 
  porque\footnote{Las derivadas de $y$ pueden comprobarse como ejercicio para el lector.}
  \begin{align*}
    y^{\prime} &= e^x (5 \cos (2x) - 5 \sin (2x)) \\[0.5em]
    y^{\prime \prime} &= e^x (-5 \cos (2x) - 15 \sin (2x))
  \end{align*}
  Sustituyendo en la \ed :
  \[(e^x(-5 \cos (2x) -15 \sin (2x)) - 2(e^x(5 \cos (2x) -5 \sin (2x)) 
  +5(e^x(3 \cos (2x) + \sin (2x))) = 0\]
  Si $A = e^x \cos(2x)$ y $B = e^x \sin(2x)$ entonces
  \begin{gather*}
    -5A - 15B -10A + 10B +15A + 5B = 0\\
    -15A + 15A -15B + 15 B = 0\\
    \therefore 0 = 0
  \end{gather*}
\end{ejemplo}

Procedamos entonces a realizar los ejercicios correspondientes a la \cref{tarea:soluciones}.

\begin{tarea}\label{tarea:soluciones}
  Demostrar si las siguientes funciones son soluciones de la correspondiente ecuación diferencial.
  
  \begin{ejercicio}[2]
  $y = 2e^{-2x} + \frac{1}{3} e^x$ de $y^{\prime} + 2y = e^x$
    \[ y^{\prime} = -4e^{-2x} + \frac{1}{3} e^x \]
    Sustituyendo en la \ed:
    \begin{gather*}
      -4e^{-2x} + \frac{1}{3} e^x + 2(2e^{-2x} + \frac{1}{3} e^x) = e^x\\
      -4e^{2x} + \frac{1}{3} e^x + 4e^{-2x} + \frac{2}{3} e^x = e^x\\
      \cancel{-4e^{-2x}} + e^x \cancel{+ 4e^{-2x}} = e^x\\
      \therefore e^x = e^x
    \end{gather*}
  \end{ejercicio}
  \begin{ejercicio}[4]
    $y = C_1 e^{-x} + C_2 e^{2x}$ de $y^{\prime \prime} - y^{\prime} -2y = 0$
    \begin{gather*}
      y^{\prime} = -C_1 e^{-x} + 2C_2e^{2x}\\[0.5em]
      y^{\prime \prime} = C_1e^{-x} + 4C_2e^{2x}
    \end{gather*}
    Sustituyendo en la \ed :
    \[ C_1e^{-x} + 4C_2e^{2x} - (-C_1e^{-x} + 2C_2e^{2x}) - 2(C_1e^{-x} + C_2e^{2x}) = 0 \]
    \[ C_1e^{-x} + 4C_2e^{2x} + C_1e^{-x} - 2C_2e^{2x} - 2C_1e^{-x} -2C_2e^{2x} = 0 \]
    \[ \therefore 0 = 0 \]
  \end{ejercicio}
  \begin{ejercicio}[5]
    $y = 8e^x + xe^x$ de $y^{\prime \prime} - 2y^{\prime} + y = 0$
    \begin{gather*}
      y^{\prime} = 9e^x + xe^x\\[0.5em]
      y^{\prime \prime} = 10e^x + xe^x
    \end{gather*}
    Sustituyendo en la \ed :
    \[ 10e^x + xe^x - 2(9e^x + xe^x) + 8e^x + xe^x = 0 \]
    \[ 10e^x + xe^x - 18e^x - 2xe^x + 8e^x + xe^x = 0 \]
    \[ \therefore 0 = 0 \]
  \end{ejercicio}
  \begin{ejercicio}[7]
    $y = \frac{1}{\cos(x)}$ de $y^{\prime} - y \tan(x) = 0$
    \begin{gather*}
      y = \sec(x)\\
      y^{\prime} = \tan(x) \cdot \sec(x)
    \end{gather*}
    Sustituyendo en la \ed :
    \[ (\tan(x) \cdot \sec(x)) - (\tan(x) \cdot \sec(x)) = 0\]
    \[ \therefore 0 = 0 \]
  \end{ejercicio}
  \begin{ejercicio}[9]
    $y = 5 \tan(5x)$ de $y^{\prime} = 25 + y^2$
    \[ y^{\prime} = 25 \sec^2(5x) \]
    Sustituyendo en la \ed :
    \begin{gather*}
      25 \sec^2(5x) = 25 + (5 \tan(5x))^2\\[0.5em]
      25 \sec^2(5x) = 25 + 25 \tan^2(5x)\\[0.5em]
      25 \sec^2(5x) - 25 \tan^2(5x) = 25\\[0.5em]
      25(\sec^2(5x) - \tan^2(5x)) = 25
    \end{gather*}
    Por la identidad trigonométrica de $\tan^2(ax) + 1 = \sec^2(ax)$
    \[ \therefore 25 = 25 \]
  \end{ejercicio}
  \begin{ejercicio}[10]
    $y = (\sqrt{x} + c)^2$ de $y^{\prime} = \sqrt{\frac{y}{x}}$
    \[ y^{\prime} = \frac{\sqrt{x} + c}{\sqrt{x}} \]
    Sustituyendo en la \ed :
    \begin{gather*}
      1 + \frac{c}{\sqrt{x}} = \sqrt{\frac{(\sqrt{x}+c)^2}{x}}\\[0.5em]
      1 + \frac{c}{\sqrt{x}} = \frac{\sqrt{x} + c}{\sqrt{x}}\\[0.5em]
      \therefore 1 + \frac{c}{\sqrt{x}} = 1 + \frac{c}{\sqrt{x}}
    \end{gather*}
  \end{ejercicio}
\end{tarea}

\section{Familia de soluciones n-paramétricas}
Una solución que contiene una o varias constantes arbitrarias representa un conjunto de soluciones
llamado ``familia de soluciones n-paramétricas''. Esto significa que una sola
ecuación diferencial puede tener un número infinito de soluciones que
corresponden a un número limitado de elecciones de los parámetros.
\begin{ejemplo}
  Sea $y = C_1e^x + C_2xe^x$ una familia de soluciones de la \ed 
 
  $y^{\prime \prime} - 2y^{\prime} + y = 0$

  Ejemplos de soluciones particulares:
  \begin{itemize}
    \item Si $C_1 = 0$ y $C_2 = 0$ entonces\footnote{$y_p$ es de solución particular}
$y_p = 0$ (solución trivial)
    \item Si $C_1 = 1$ y  $C_2 = 0$ entonces $y_p = e^x$
    \item Si $C_1 = -2$ y $C_2 = -2/3$ entonces $y_p = -2e^x + \frac{2}{3} xe^x$
  \end{itemize}
\end{ejemplo}
\begin{ejemplo}
Si $x(t) = C_1 \cos(4t) + C_2 \sin(4t)$ es una familia de soluciones de la \ed :
$x^{\prime \prime} + 16x = 0$. Determinar una solución particular que satisfaga las condiciones iniciales:
$x(\frac{\pi}{2}) = -2$ y $x^{\prime}(\frac{\pi}{2}) = 1$.

Aplicando condiciones iniciales:
\begin{gather*}
  x(\frac{\pi}{2}) = C_1 \cos(4(\frac{\pi}{2})) + C_2 \sin(4(\frac{\pi}{2}))\\[0.5em]
  = C_1 \cos(2 \pi) + C_2 \sin(2 \pi)\\[0.5em]
  = C_1(1) + C_2(0)\\[0.5em]
  =C_1 + 0\\[0.5em]
  C_1 = -2
\end{gather*}
\begin{gather*}
  x^{\prime}(t) = -4 C_1 \sin(4t) + 4 C_2 \cos(4t)\\[0.5em]
  x^{\prime}(\frac{\pi}{2}) = -4 C_1 \sin(4(\frac{\pi}{2})) + 4 C_2 \cos(4(\frac{\pi}{2}))\\[0.5em]
= -4(-2) \cancel{\sin(2 \pi)} + 4 C_2 \cos(2 \pi)\\[0.5em]
= 4 C_2 \cos(2 \pi)\\[0.5em]
= 4 C_2 (1)\\[0.5em]
1 = 4 C_2\\[0.5em]
C_2 = \frac{1}{4}
\end{gather*}
Entonces una solución particular es con $C_1 = -2$ y $C_2 = \frac{1}{4}$
\[ \therefore y_p = -2 \cos(4t) + \frac{1}{4} \sin(4t) \]
\end{ejemplo}
\begin{ejemplo}
  De la familia de soluciones $y_g = C_1 e^x + C_2 e^{-x}$ obtenga las
  soluciones particulares que satisfagan $y(-1) = 5$ y $y^{\prime}(-1) = -5$.
  \begin{gather*}
    y(-1) = C_1 e^{-1} + C_2 e^1 = 5\\
    = \frac{C_1}{e} + C_2 e = 5
  \end{gather*}
  \begin{gather*}
    y^{\prime} = C_1 e^x - C_2 e^{-x}\\[0.5em]
    y^{\prime}(-1) = C_1 e^{-1} - C_2 e^1 = -5\\[0.5em]
    \frac{C_1}{e} - C_2 e = -5
  \end{gather*}
  Resolvemos el sistema de ecuaciones
\[
\begin{array}{r@{\,}c@{\,}l}
  \frac{C_1}{e} & + C_2 e & = 5 \\[0.5em]
  \frac{C_1}{e} & - C_2 e & = -5 \\[0.5em]
  \hline
  \rule{0pt}{1.2em}2\frac{C_1}{e} &         & = 0
\end{array}
\]
\[ \therefore C_1 = 0 \]
Entonces 
\begin{gather*}
0 + C_2 e = 5\\
\therefore C_2 = \frac{5}{e}
\end{gather*}
Finalmente
\begin{gather*}
  y_p = \frac{5}{e} e^{-x} = 5 e^{-1} e^{-x} = 5e^{-(x+1)}\\
  y_p = 5e^{-(x+1)}
\end{gather*}

Ahora, para las \textbf{condiciones iniciales} $y(0) = 1$ y $y^{\prime}(0) = 2$
\begin{gather*}
  y(0) = C_1 e^0 + C_2 e^{-0} = 1\\[0.5em]
  = C_1(1)+C_2(1) = 1\\[0.5em]
  = C_1 + C_2 = 1
\end{gather*}
\begin{gather*}
  y^{\prime} = C_1 e^x - C_2 e^{-x}\\[0.5em]
  y^{\prime}(0) = C_1 e^0 - C_2e^{-0} = 2\\[0.5em]
  = C_1(1) - C_2(1) = 2\\[0.5em]
  C_1 - C_2 = 2
\end{gather*}
Resolvemos el sistema de ecuaciones
\[
  \begin{array}{rcl}
    C_1 & + C_2 & = 1\\
    C_1 & - C_2 & = 2\\
    \hline
    2C_1 & & = 3
  \end{array}
\]
\[ \therefore C_1 = \frac{3}{2} \]
\begin{gather*}
  \frac{3}{2} - C_2 = 2\\
  -C_2 = \frac{4}{2} - \frac{3}{2}\\
  -C_2 = \frac{1}{2}\\
  \therefore C_2 = - \frac{1}{2}
\end{gather*}
\[ y_p = \frac{3}{2}e^x - \frac{1}{2}e^{-x} \]
\end{ejemplo}
Procedamos entonces a realizar los ejercicios correspondientes a la \cref{tarea:familiaSoluciones}.
\begin{tarea}\label{tarea:familiaSoluciones}
De la familia de soluciones $x(t) = C_1 \cos(t) + C_2 \sin(t)$ obtenga las 
soluciones particulares que satisfacen las condiciones iniciales.

Si tenemos \[ x(t) = C_1 \cos(t) + C_2 \sin(t) \]
Entonces \[ x^{\prime}(t) = -C_1 \sin(t) + C_2 \cos(t) \]
\begin{ejercicio}
  $x(0) = -1$ y $x^{\prime}(0) = 8$

  Evaluamos $x(0)$
  \begin{gather*}
    x(0) = C_1 \cos(0) + C_2 \sin(0)\\[0.5em]
    = C_1(1) + C_2(0)\\[0.5em]
    C_1  = 1
  \end{gather*}

  Evaluamos $x^{\prime}(0)$
  \begin{gather*}
    x^{\prime}(0) = -C_1 \sin(0) + C_2 \cos(0)\\[0.5em]
    = -C_1(0) + C_2(1)\\[0.5em]
    C_2 = 8
  \end{gather*}
  Solución particular: $x(t) = - \cos(t) + 8 \sin(t)$

\end{ejercicio}
\begin{ejercicio}
  $x(\frac{\pi}{4}) = \sqrt{2}$ y $x^{\prime}(\frac{\pi}{4}) = 2 \sqrt{2}$

  Evaluamos $x(\frac{\pi}{4})$
  \begin{gather*}
    x(\frac{\pi}{4}) = C_1 \cos(\frac{\pi}{4}) + C_2 \sin(\frac{\pi}{4})\\[0.5em]
    = C_1 (\frac{\sqrt{2}}{2}) + C_2 (\frac{\sqrt{2}}{2}) = \sqrt{2}\\[0.5em]
    = \frac{\sqrt{2}}{2}(C_1 + C_2) = \sqrt{2}\\[0.5em]
    \sqrt{2}(C_1 + C_2) = 2 \sqrt{2}\\[0.5em]
    C_1 + C_2 = 2
  \end{gather*}

  Evaluamos $x^{\prime}(\frac{\pi}{4})$
  \begin{gather*}
    x^{\prime}(\frac{\pi}{4}) = -C_1 \sin(\frac{\pi}{4}) + C_2 \cos(\frac{\pi}{4}) = 2 \sqrt{2}\\[0.5em]
    -C_1 (\frac{\sqrt{2}}{2}) + C_2 (\frac{\sqrt{2}}{2}) = 2 \sqrt{2}\\[0.5em]
    \frac{\sqrt{2}}{2}(C_2 - C_1) = 2 \sqrt{2}\\[0.5em]
    \sqrt{2}(C_2 - C_1) = 4 \sqrt{2}\\[0.5em]
    C_2 - C_1 = 4
  \end{gather*}

  Resolvemos el sistema de ecuaciones
  \[
  \begin{array}{rcl}
    C_1 & + C_2 & = 2\\
    -C_1 & + C_2 & = 4\\
    \hline
         & 2C_2 & = 6
  \end{array}
  \]
  \begin{gather*}
    C_2 = 6/2\\
    C_2 = 3
  \end{gather*}
  \begin{gather*}
    C_1 + 3 = 2\\
    C_1 = 2 - 3\\
    C_1 = -1
  \end{gather*}
  Solución particular: $x(t) = - \cos(t) + 3 \sin(t)$
\end{ejercicio}
\end{tarea}

\section{Primitivas}
La primitiva es la ecuación diferencial de la cual se origina la solución general.
\begin{ejemplo}
  Hallar la primitiva a partir de la solución general $y_g = x + cx^2$

  Derivando la solución general se obtiene 
  \begin{equation}
    y^{\prime} = 1 + 2cx
    \label{eq:ejemploPrimitivas}
  \end{equation}
  Despejamos $c$ de la solución general dada y la sustituimos en la derivada obtenida en \ref{eq:ejemploPrimitivas}
  \[ c = \frac{y - x}{x^2} \]
  Se obtiene la \ed de primer orden \[ y^{\prime} = 1 + 2 \left( \frac{y-x}{x} \right) \]
\end{ejemplo}
\textbf{Tip:} Se deberá derivar la solución general tantas veces como esta contenga constantes.
\begin{ejemplo}
  Hallar la primitiva a partir de la solución general $y = ax^2 + bx + c$

  Dado que hay tres constantes en la solución general, se deriva tres veces, obteniendo
  \begin{gather*}
    y^{\prime} = 2ax + b\\
    y^{\prime \prime} = 2a\\
    y^{\prime \prime \prime} = 0
  \end{gather*}
  La primitiva es la \ed de tercer orden \[ y^{\prime \prime \prime} = 0\]
\end{ejemplo}
\begin{itemize}
  \item Cada constante arbitraria necesita una derivada para eliminarse.
  \item Te detienes cuando \textbf{todas las constantes desaparecen}.
  \item El \textbf{orden} de la ecuación diferencial es quivalente al \textbf{numero de constantes} en la solución general.
  \item La ecuación diferencial que resulta es \textbf{la más simple} que "genera" todas las soluciones posibles.
\end{itemize}
\begin{ejemplo}
Hallar la primitiva a partir de la solución general $cx^2 + y^2 = 1$

Derivando de forma implícita la solución general \[ 2cx + 2yy^{\prime} = 0 \]
Despejamos $c$ de la solución general dada y la sustituimos en la derivada obtenida
\begin{gather*}
  cx^2 = 1 - y^2\\
  c = \frac{1 - y^2}{x^2}
\end{gather*}
\begin{gather*}
  \left( \frac{1 - y^2}{x^2} \right) x + yy^{\prime} = 0\\[0.5em]
  yy^{\prime} = - \left( \frac{1 - y^2}{x^2} \right) x\\[0.5em]
  y^{\prime} = - \frac{1}{y} \left( \frac{1 - y^2}{x} \right)
\end{gather*}
Por lo tanto, la primitiva es la \ed de primer orden \[ y^{\prime} = \frac{y^2 - 1}{xy} \]
\end{ejemplo}
\begin{ejemplo}
  Hallar la primitiva a partir de la solución general $y = C_1x + C_2x^3$

  Derivando dos veces la solución general se obtiene
  \begin{gather*}
    y^{\prime} = C_1 + 3C_2x^2\\[0.5em]
    y^{\prime \prime} = 6C_2x
  \end{gather*}
  Resolvemos el sistema de ecuaciones para $C_1$ y $C_2$ a partir de las derivadas
  \begin{gather*}
    C_2 = \frac{y^{\prime \prime}}{6x}\\[0.5em]
    y^{\prime} = C_1 + 3 \left( \frac{y^{\prime \prime}}{6x} \right) x^2\\[0.5em]
    y^{\prime} = C_1 + \frac{1}{2} \left( \frac{y^{\prime \prime}}{x} \right) x^2\\[0.5em]
    y^{\prime} = C_1 + 1/2 y^{\prime \prime} x\\[0.5em]
    C_1 = y^{\prime} - \frac{x y^{\prime \prime}}{2}
  \end{gather*}
  Sustituyendo en la solución general
  \begin{gather*}
    y = \left( y^{\prime} - \frac{x y^{\prime \prime}}{2} \right) x + \left( \frac{y^{\prime \prime}}{6x} \right) x^3\\[1em]
    y = xy^{\prime} - \frac{x^2y^{\prime \prime}}{2} + \frac{x^2y^{\prime \prime}}{6}
  \end{gather*}
  La primitiva es la \ed de segundo orden \[ x^2y^{\prime \prime} - 3xy^{\prime} + 3y = 0 \]
\end{ejemplo}
Procedamos entonces a realizar los ejercicios correspondientes a la \cref{tarea:Primitivas}
\begin{tarea}
  \label{tarea:Primitivas}
  Obtener la ecuación diferencial a partir de la primitiva.

  \begin{ejercicio}[1]
    \label{ejercicio:primitivas:guess}
  \[ y = C_1 e^{5x} + C_2 xe^{5x} \]
  Puesto que hay dos constantes, derivamos dos veces:
  \begin{gather*}
    y^{\prime} = C_1 (5e^{5x}) + C_2 (5xe^{5x} + e^{5x})\\[0.5em]
    y^{\prime \prime} = 25C_1e^{5x} + 25C_2xe^{5x} + 10C_2e^{5x}
  \end{gather*}
  Resolvemos el sistema de ecuaciones para $C_1$ y $C_2$ a partir de las derivadas.

  Agrupas los términos con $e^{5x}$ y los términos con $xe^{5x}$
  \begin{gather*}
    y^{\prime} = e^{5x} (5C_1 + C_2) + 5C_2xe^{5x}\\[0.5em]
    y^{\prime \prime} = e^{5x}(25C_1 + 10C_2) + 25C_2xe^{5x}
  \end{gather*}
  \textbf{Tip:} Agrupa los "bloques base". Como en este caso lo son los términos con $e^{5x}$ y los términos con $xe^{5x}$

  Este formato es muy útil porque facilita que posteriormente los coeficientes de $e^{5x}$ y de $xe^{5x}$ se cancelen entre sí.

  Observa cómo crecen los coeficientes de $e^{5x}$ y $xe^{5x}$
  \begin{gather*}
    y = C_1e^{5x} + C_2xe^{5x}\\
    y^{\prime} = e^{5x} (5C_1 + C_2) + 5 C_2 xe^{5x}\\[0.5em]
    y^{\prime \prime} = e^{5x} (25C_1 + 10C_2) + 25C_2xe^{5x}
  \end{gather*}
  Parecen múltiplos de 5 y 25, así que podemos intentar algo como \[ y^{\prime \prime} - ay^{\prime} + by = 0 \]
  Lo más intuitivo es probar con $a = 10$ y $b = 25$ para intentar empatar los múltiplos, porque 
  $y^{\prime \prime} \rightarrow 25C_1 + 10C_2$ contra $10y^{\prime} \rightarrow 10(5C_1 + C_2)$
  \[ \therefore y^{\prime \prime} - 10y^{\prime} + 25y = 0 \]
  \end{ejercicio}
  \begin{ejercicio}[2]
    \[ y = \frac{1}{3} e^{-3x} + C \]
    Como hay una constante, derivamos una vez
    \begin{gather*}
      y^{\prime} = \frac{1}{3} (-3e^{-3x}) = -e^{-3x}\\
      y^{\prime} = -e^{-3x}\\
      \therefore y^{\prime} + e^{-3x} = 0
    \end{gather*}
    Si derivas la solución y te da algo explícito en $x$ (sin $y$'s), entonces lo más probable es que la \ed sea 
    de la forma $y^{\prime} + f(x) = 0$
  \end{ejercicio}
  
  \begin{ejercicio}[4]
    \[ y = C_1e^{-x} + C_2e^{2x} \]
    Como hay dos constantes, derivamos dos veces
    \begin{gather*}
      y^{\prime} = -C_1e^{-x} + 2C_2e^{2x}\\
      y^{\prime \prime} = C_1e^{-x} + 4C_2e^{2x}
    \end{gather*}
    A partir del método utilizado en el \cref{ejercicio:primitivas:guess}, deducimos que la \ed tendrá la forma 
    $y^{\prime \prime} + ay^{\prime} + by = 0$

    Podemos entonces introducir un método más general para la obtención de la ecuación diferencial a partir 
    de la solución general.

    Cuando una solución es de la forma \[ y = C_1e^{r_1x} + C_2e^{r_2x} \]
    la ecuación diferencial asociada es \[ y - (r_1 + r_2)y^{\prime} + (r_1 \cdot r_2)y = 0 \]

    En este caso, $r_1 = -1$ y $r_2 = 2$. Calculamos 
    \begin{gather*}
      r_1 + r_2 = -1 + 2 = 1\\
      r_1 \cdot r_2 = (-1)(2) = -2
    \end{gather*}
    \begin{gather*}
      y^{\prime \prime} -(1)y^{\prime} -2y = 0\\
      \therefore y^{\prime \prime} - y^{\prime} - 2y = 0
    \end{gather*}
  \end{ejercicio}
  \begin{ejercicio}[7]
    \[ y = x + 5 \ln(x+1) + C \]
    Como hay una constante, derivamos una vez
    \begin{gather*}
      y^{\prime} = 1 + \frac{5}{x+1}\\[0.5em]
      = \frac{x+1}{x+1} + \frac{5}{x+1}\\[0.5em]
      = \frac{x+1+5}{x+1}\\[0.5em]
      = \frac{x+6}{x+1}
    \end{gather*}
    \[ \therefore y^{\prime} = \frac{x+6}{x+1} \]
  \end{ejercicio}
  \begin{ejercicio}[9]
    \[ C = \frac{x^4}{4} + xy^{3} \]
    Derivamos implícitamente \[ 0 = x^3 + 3xy^2y^{\prime} + y^3 \]
    Despejamos $y^{\prime}$
    \begin{gather*}
      -x^3 - y^3 = 3xy^2y^{\prime}\\[0.5em]
      y^{\prime} = \frac{-x^3-y^3}{3xy^2}\\[0.5em]
      \therefore y^{\prime} = - \frac{x^3 + y^3}{3xy^2}
    \end{gather*}
  \end{ejercicio}
\end{tarea}
A partir de la Sección \ref{sec:variablesSeparables} estudiaremos métodos para \textbf{resolver} \ed .

\section{Ecuaciones diferenciales de variables separables.}
\label{sec:variablesSeparables}
Una ecuación de variables separables tiene la forma hola buenas tardes, esto es una prueba para el 
nuevo script de GitHub Actions 2
\end{document}
