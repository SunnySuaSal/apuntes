\documentclass[12pt]{article}

\usepackage[utf8]{inputenc} %Codificación de caracteres (UTF-8)
\usepackage{amsmath, amssymb, amsfonts, cancel} %Paquetes para expresiones matemáticas
\usepackage{graphicx} %Para insertar imágenes
\usepackage[spanish]{babel} %Optimiza el typesetting para documentos en español
\usepackage[letterpaper, left=1in, right=1in, top=1in, bottom=1in]{geometry}
%Para abarcar más hoja horizontalmente
\usepackage{booktabs} %Optimiza trabajar con tablas y agrega algunos comandos
\usepackage{multirow} %Para crear celdas tabulares que abarcan múltiples filas
\usepackage{float} %Mayor control sobre dónde se colocan las figuras y tablas
\usepackage{caption} %Mayor control de las captions de figuras y tablas
\usepackage{colortbl} %Para colores en tablas
\usepackage{xcolor} %Opcional, pero útil para definir colores personalizados
\definecolor{paleYellow}{RGB}{255, 255, 180} %Ejemplo con amarillo pálido
\usepackage{physics} %Notación elegante de derivadas (opcional)
\usepackage{hyperref}

\title{Asesorías Cálculo A}
\author{Jorge Alberto Suárez Saldaña}
\date{\today}

\begin{document}
\maketitle
En las presentes notas de las asesorías, recorreremos los temas listados en el temario del curso. Tomando un enfoque más práctico con una ligera explicación teórica, 
la resolución de al menos un ejercicio y la asignación de varios ejercicios de práctica.
\section{Definición intuitiva de límite}
El límite de una función en un punto describe a qué valor se acerca la función cuando x se aproxima a ese punto, aunque la función no esté definida allí.
Significa acercarce mucho a un valor sin llegar a este.
\subsection{Ejemplo}
La función $f(x)=\frac{x^2-1}{x-1}$ no se puede evaluar en $x=1$, pues se indefine (da $0/0$), pero si evaluamos en valores cercanos:
\begin{table}[H]
\centering
\begin{tabular}{cc}
$x$&$f(x)$\\
0.9&1.9\\
0.99&1.99\\
1.01&2.01\\
1.1&2.1
\end{tabular}
\end{table}
Observamos que la función se acerca a 2, por lo tanto \[ \lim_{x \to 1} f(x) = 2 \]
La definición intuitiva formal es la siguiente: \[ \lim_{x \to a} f(x)=L \]
Que se lee como: El límite de $f(x)$ cuando x tiende a $a$ es $L$.
\subsection{Ejercicios (tabulación)}
\begin{enumerate}
  \item \[ \lim_{x \to 2} (x^2+1) \]
  \item \[ \lim_{x \to 0} \frac{\sin(x)}{x} \]
  \item \[ \lim_{x \to 1} \frac{|x-1|}{x-1} \]
\end{enumerate}
\section{Análisis gráfico de límites}
Podemos examinar los límites de una función si contamos con la gráfica de dicha función, 
especialmente si utilizamos un graficador digital como \textit{Desmos} o \textit{GeoGebra}.
El procedimiento es el siguiente:
\begin{enumerate}
  \item Observa la gráfica.
  \item Observa qué valor toma $f(x)$ cuando te acercas por izquierda y popr derecha.
  \item Si \underline{ambos} coinciden, el límite existe.
\end{enumerate}
Casos importantes en gráficas:
\begin{enumerate}
  \item \textbf{Hueco:} La gráfica se acerca a un punto, pero no está definida allí. Entonces sí hay límite. Ejemplo en la Figura \ref{fig:limitePunto}.
  \item \textbf{Salto:} Valores distintos por izquierda y derecha. Entonces el límite no existe. Ejemplo en la Figura \ref{fig:limiteSalto}.
  \item \textbf{Asíntota vertical:} La gráfica se dispara. Entonces el límite es infinito. Ejemplo en la Figura \ref{fig:limiteAsintota}.
\end{enumerate}
\begin{figure}
  \begin{center}
    \includegraphics[width=0.6\textwidth]{assets/limitePunto.jpeg}
  \end{center}
  \caption{Ejemplo de límite hueco.}\label{fig:limitePunto}
\end{figure}
\begin{figure}
  \begin{center}
    \includegraphics[width=0.6\textwidth]{assets/limiteSalto.jpeg}
  \end{center}
  \caption{Ejemplo de límite con salto.}\label{fig:limiteSalto}
\end{figure}
\begin{figure}
  \begin{center}
    \includegraphics[width=0.6\textwidth]{assets/limiteAsintota.jpeg}
  \end{center}
  \caption{Ejemplo de límite con asíntota.}\label{fig:limiteAsintota}
\end{figure}

\section{La derivada y el problema de la recta tangente}

La derivada de una función surge del problema de encontrar la pendiente de la recta tangente a una curva en un punto dado.
Recordemos que la pendiente de una recta secante que pasa por los puntos $(a,f(a))$ y $(a+h,f(a+h))$ está dada por:
\[
m_{\text{secante}}=\frac{f(a+h)-f(a)}{h}
\]
Cuando hacemos que $h$ tienda a cero, la recta secante se convierte en la recta tangente.

\subsection{Definición de la derivada}
La derivada de una función $f$ en el punto $a$ se define como:
\[
f'(a)=\lim_{h\to 0}\frac{f(a+h)-f(a)}{h}
\]
Esta expresión representa:
\begin{itemize}
  \item La pendiente de la recta tangente a la gráfica de $f$ en $x=a$.
  \item La razón de cambio instantánea de la función.
\end{itemize}

\subsection{Ejemplo}
Sea $f(x)=x^2$. Calculamos su derivada usando la definición:
\[
f'(a)=\lim_{h\to0}\frac{(a+h)^2-a^2}{h}
\]
\[
=\lim_{h\to0}\frac{a^2+2ah+h^2-a^2}{h}
=\lim_{h\to0}(2a+h)=2a
\]
Por lo tanto:
\[
f'(x)=2x
\]

\subsection{Recta tangente}
La ecuación de la recta tangente a $f(x)$ en el punto $x=a$ es:
\[
y-f(a)=f'(a)(x-a)
\]

\subsection{Ejercicios}
\begin{enumerate}
  \item Calcula $f'(a)$ para $f(x)=x^3$ usando la definición.
  \item Encuentra la ecuación de la recta tangente a $f(x)=x^2$ en $x=2$.
  \item Calcula la pendiente de la recta tangente a $f(x)=\sqrt{x}$ en $x=4$.
\end{enumerate}

%--------------------------------------------------

\section{Forma alternativa de la derivada}

Una forma equivalente de definir la derivada es:
\[
f'(a)=\lim_{x\to a}\frac{f(x)-f(a)}{x-a}
\]
Esta expresión es especialmente útil para el cálculo práctico de derivadas en un punto.

\subsection{Ejemplo}
Sea $f(x)=x^2$ y $a=3$:
\[
f'(3)=\lim_{x\to3}\frac{x^2-9}{x-3}
\]
Factorizando:
\[
=\lim_{x\to3}\frac{(x-3)(x+3)}{x-3}=6
\]

\subsection{Ejercicios}
\begin{enumerate}
  \item Calcula $f'(1)$ para $f(x)=x^2$ usando esta forma.
  \item Calcula $f'(2)$ para $f(x)=\frac{1}{x}$.
  \item Analiza si existe la derivada de $f(x)=|x|$ en $x=0$.
\end{enumerate}

%--------------------------------------------------

\section{Velocidad, aceleración y razones de cambio}

La derivada permite interpretar fenómenos físicos y reales como razones de cambio.

\subsection{Velocidad}
Si $s(t)$ representa la posición de un objeto en función del tiempo, la velocidad instantánea está dada por:
\[
v(t)=s'(t)
\]

\subsection{Ejemplo}
Sea $s(t)=t^2$. Entonces:
\[
v(t)=2t
\]
En $t=3$, la velocidad es:
\[
v(3)=6
\]

\subsection{Aceleración}
La aceleración es la derivada de la velocidad, o equivalentemente, la segunda derivada de la posición:
\[
a(t)=v'(t)=s''(t)
\]
En el ejemplo anterior:
\[
a(t)=2
\]

\subsection{Otras razones de cambio}
La derivada también se usa para modelar:
\begin{itemize}
  \item Crecimiento poblacional
  \item Cambio de temperatura
  \item Costo e ingreso marginal
  \item Cambio de volumen o área
\end{itemize}

\subsection{Ejemplo}
Si el costo de producir $x$ unidades está dado por:
\[
C(x)=x^2+3x
\]
El costo marginal es:
\[
C'(x)=2x+3
\]

\subsection{Ejercicios}
\begin{enumerate}
  \item Dada $s(t)=3t^2-2t$, calcula la velocidad y la aceleración.
  \item Si $T(t)=20+5t$, interpreta $T'(t)$.
  \item Si el radio de una esfera cambia como $r(t)=2t$, ¿a qué razón cambia el radio?
\end{enumerate}

\section{Teorema de Rolle}

El Teorema de Rolle establece la existencia de al menos un punto donde la recta tangente a la gráfica de una función es horizontal, siempre que se cumplan ciertas condiciones.

\subsection{Enunciado}
Sea $f(x)$ una función que cumple:
\begin{enumerate}
  \item Es continua en el intervalo cerrado $[a,b]$
  \item Es derivable en el intervalo abierto $(a,b)$
  \item $f(a)=f(b)$
\end{enumerate}
Entonces existe al menos un punto $c\in(a,b)$ tal que:
\[
f'(c)=0
\]

\subsection{Interpretación geométrica}
Si la función comienza y termina en el mismo valor, y no tiene saltos ni picos, entonces en algún punto intermedio la pendiente de la tangente debe ser cero.

\subsection{Ejemplo}
Sea:
\[
f(x)=x^2-4x+3 \quad \text{en } [1,3]
\]
Verificamos:
\begin{itemize}
  \item Es un polinomio, por lo tanto continua y derivable.
  \item $f(1)=0$ y $f(3)=0$
\end{itemize}

Derivamos:
\[
f'(x)=2x-4
\]
Buscamos $f'(c)=0$:
\[
2x-4=0 \Rightarrow x=2
\]

\subsection{Ejercicios}
\begin{enumerate}
  \item Verifica si se puede aplicar el Teorema de Rolle a $f(x)=x^3$ en $[-1,1]$.
  \item Encuentra el valor de $c$ para $f(x)=x^2-1$ en $[-1,1]$.
\end{enumerate}

%--------------------------------------------------

\section{Teorema del Valor Medio}

El Teorema del Valor Medio generaliza el Teorema de Rolle y relaciona la pendiente promedio con la pendiente instantánea.

\subsection{Enunciado}
Sea $f(x)$ una función que cumple:
\begin{enumerate}
  \item Es continua en $[a,b]$
  \item Es derivable en $(a,b)$
\end{enumerate}
Entonces existe al menos un punto $c\in(a,b)$ tal que:
\[
f'(c)=\frac{f(b)-f(a)}{b-a}
\]

\subsection{Interpretación}
La pendiente de la recta secante entre $(a,f(a))$ y $(b,f(b))$ coincide con la pendiente de la recta tangente en algún punto intermedio.

\subsection{Ejemplo}
Sea $f(x)=x^2$ en $[1,3]$.
\[
\frac{f(3)-f(1)}{3-1}=\frac{9-1}{2}=4
\]
Derivando:
\[
f'(x)=2x
\]
\[
2x=4 \Rightarrow x=2
\]

\subsection{Ejercicios}
\begin{enumerate}
  \item Aplica el Teorema del Valor Medio a $f(x)=x^3$ en $[0,2]$.
  \item Interpreta el Teorema del Valor Medio en términos de velocidad promedio e instantánea.
\end{enumerate}

%--------------------------------------------------

\section{Problemas de optimización}

Los problemas de optimización buscan maximizar o minimizar una cantidad bajo ciertas restricciones.

\subsection{Estrategia general}
\begin{enumerate}
  \item Definir las variables.
  \item Escribir la función que se desea optimizar.
  \item Calcular la derivada.
  \item Resolver $f'(x)=0$.
  \item Analizar el resultado y dar una interpretación.
\end{enumerate}

\subsection{Ejemplo}
Un rectángulo tiene perímetro 20. Determinar las dimensiones que maximizan el área.

Sea $x$ el ancho y $y$ el largo:
\[
2x+2y=20 \Rightarrow y=10-x
\]
Área:
\[
A(x)=x(10-x)=10x-x^2
\]
Derivamos:
\[
A'(x)=10-2x
\]
\[
10-2x=0 \Rightarrow x=5
\]
Entonces:
\[
y=5
\]

\subsection{Conclusión}
El área máxima se obtiene cuando el rectángulo es un cuadrado.


\subsection{Ejercicios}
\begin{enumerate}
  \item Un rectángulo tiene perímetro $P=30$. Determina las dimensiones que maximizan su área.
  \item Una lata cilíndrica debe tener volumen $V=1000$. Determina el radio y la altura que minimizan el área total.
\end{enumerate}

%--------------------------------------------------

\section{Regla de L'Hôpital}

La Regla de L'Hôpital se utiliza para evaluar límites que presentan ciertas indeterminaciones.

\subsection{Indeterminaciones permitidas}
\[
\frac{0}{0}, \quad \frac{\infty}{\infty}
\]

\subsection{Enunciado}
Si:
\[
\lim_{x\to a}\frac{f(x)}{g(x)}=\frac{0}{0}
\quad \text{o} \quad
\frac{\infty}{\infty}
\]
y las derivadas existen, entonces:
\[
\lim_{x\to a}\frac{f(x)}{g(x)}=
\lim_{x\to a}\frac{f'(x)}{g'(x)}
\]

\subsection{Ejemplo}
\[
\lim_{x\to0}\frac{\sin x}{x}
\]
Derivando numerador y denominador:
\[
\lim_{x\to0}\frac{\cos x}{1}=1
\]

\subsection{Ejemplo al infinito}
\[
\lim_{x\to\infty}\frac{\ln x}{x}
\]
\[
\lim_{x\to\infty}\frac{1/x}{1}=0
\]

\subsection{Advertencias}
\begin{itemize}
  \item Solo se aplica si existe una indeterminación válida.
  \item Se debe derivar numerador y denominador.
  \item Puede aplicarse más de una vez si es necesario.
\end{itemize}

\subsection{Ejercicios}
\begin{enumerate}
  \item $\displaystyle \lim_{x\to0}\frac{e^x-1}{x}$
  \item $\displaystyle \lim_{x\to\infty}\frac{x^2}{e^x}$
  \item $\displaystyle \lim_{x\to0}\frac{1-\cos x}{x^2}$
\end{enumerate}

\section{Integrales indefinidas}

La integral indefinida es el proceso inverso de la derivada.  
Si una función $F(x)$ cumple que
\[
F'(x)=f(x),
\]
entonces
\[
\int f(x)\,dx = F(x)+C,
\]
donde $C$ es la \textbf{constante de integración}, que aparece porque la derivada de una constante es cero.

\subsection{Ejemplo}
\[
\int 2x\,dx = x^2 + C
\]
ya que
\[
\frac{d}{dx}(x^2)=2x.
\]

\subsection{Ejercicios}
\begin{enumerate}
  \item $\displaystyle \int 3x^2\,dx$
  \item $\displaystyle \int (5x-1)\,dx$
  \item $\displaystyle \int \cos x\,dx$
\end{enumerate}

%------------------------------------------------

\section{Integrales elementales}

Las integrales elementales se resuelven aplicando directamente fórmulas conocidas.

\subsection{Regla de la potencia}
\[
\int x^n\,dx=\frac{x^{n+1}}{n+1}+C,\quad n\neq -1.
\]

\subsection{Ejemplos}
\[
\int x^3\,dx=\frac{x^4}{4}+C
\]
\[
\int \sqrt{x}\,dx=\int x^{1/2}\,dx=\frac{2}{3}x^{3/2}+C
\]

\subsection{Caso especial}
\[
\int \frac{1}{x}\,dx=\ln|x|+C
\]

\subsection{Exponenciales}
\[
\int e^x\,dx=e^x+C
\]
\[
\int a^x\,dx=\frac{a^x}{\ln a}+C
\]

\subsection{Trigonométricas básicas}
\[
\int \cos x\,dx=\sin x+C
\]
\[
\int \sin x\,dx=-\cos x+C
\]
\[
\int \sec^2 x\,dx=\tan x+C
\]

\subsection{Linealidad de la integral}
\[
\int \bigl(af(x)+bg(x)\bigr)\,dx
= a\int f(x)\,dx + b\int g(x)\,dx
\]

\subsection{Ejercicios}
\begin{enumerate}
  \item $\displaystyle \int (x^2+3x+1)\,dx$
  \item $\displaystyle \int \frac{1}{x^2}\,dx$
  \item $\displaystyle \int (e^x+\sin x)\,dx$
\end{enumerate}

%------------------------------------------------

\section{Integrales por sustitución}

La integración por sustitución se utiliza cuando la integral contiene una función compuesta y su derivada.

\subsection{Idea clave}
Si una integral tiene la forma
\[
\int f(g(x))\,g'(x)\,dx,
\]
entonces se puede simplificar usando un cambio de variable.

\subsection{Procedimiento}
\begin{enumerate}
  \item Definir $u=g(x)$.
  \item Calcular $du=g'(x)\,dx$.
  \item Sustituir en la integral.
  \item Integrar respecto a $u$.
  \item Regresar a la variable $x$.
\end{enumerate}

\subsection{Ejemplo}
\[
\int 2x\cos(x^2)\,dx
\]
Sea $u=x^2$, entonces $du=2x\,dx$:
\[
\int \cos u\,du=\sin u + C
\]
Por lo tanto,
\[
\int 2x\cos(x^2)\,dx=\sin(x^2)+C.
\]

\subsection{Otro ejemplo}
\[
\int \frac{2x}{x^2+1}\,dx
\]
Sea $u=x^2+1$, entonces $du=2x\,dx$:
\[
\int \frac{1}{u}\,du=\ln|u|+C
\]
Resultado:
\[
\ln(x^2+1)+C.
\]

\subsection{Ejercicios}
\begin{enumerate}
  \item $\displaystyle \int 3x^2 e^{x^3}\,dx$
  \item $\displaystyle \int \frac{2x}{\sqrt{x^2+1}}\,dx$
  \item $\displaystyle \int \sin(5x)\,dx$
\end{enumerate}

\section{Fórmulas básicas de integración}

A continuación se presentan las fórmulas más utilizadas en la integración de funciones elementales. 
Estas fórmulas deben memorizarse, pues son la base para resolver la mayoría de los ejercicios.

\subsection{Potencias y logaritmos}
\[
\int x^n\,dx=\frac{x^{n+1}}{n+1}+C \quad (n\neq -1)
\]
\[
\int \frac{1}{x}\,dx=\ln|x|+C
\]

\subsection{Exponenciales}
\[
\int e^x\,dx=e^x+C
\]
\[
\int a^x\,dx=\frac{a^x}{\ln a}+C
\]

\subsection{Trigonométricas}
\[
\int \sin x\,dx=-\cos x+C
\]
\[
\int \cos x\,dx=\sin x+C
\]
\[
\int \tan x\,dx=\ln|\sec x|+C
\]
\[
\int \sec^2 x\,dx=\tan x+C
\]
\[
\int \csc^2 x\,dx=-\cot x+C
\]

\subsection{Linealidad}
\[
\int \bigl(af(x)+bg(x)\bigr)\,dx
= a\int f(x)\,dx + b\int g(x)\,dx
\]

\subsection{Ejercicios}
\begin{enumerate}
  \item $\displaystyle \int (4x^3-2x+1)\,dx$
  \item $\displaystyle \int (\sin x + e^x)\,dx$
  \item $\displaystyle \int \frac{5}{x}\,dx$
\end{enumerate}

%------------------------------------------------

\section{Integración por partes}

La integración por partes se utiliza cuando la integral es un producto de dos funciones.

\subsection{Fórmula de integración por partes}
\[
\int u\,dv = uv - \int v\,du
\]

Esta fórmula se obtiene a partir de la regla del producto para derivadas.

\subsection{Elección de u}
Una regla práctica para elegir $u$ es la regla \textbf{LIATE}:
\begin{itemize}
  \item L: Logarítmicas
  \item I: Inversas
  \item A: Algebraicas
  \item T: Trigonométricas
  \item E: Exponenciales
\end{itemize}

Se elige como $u$ la función que aparece primero en la lista.

\subsection{Ejemplo}
\[
\int x e^x\,dx
\]

Elegimos:
\[
u=x \quad \Rightarrow \quad du=dx
\]
\[
dv=e^x\,dx \quad \Rightarrow \quad v=e^x
\]

Aplicando la fórmula:
\[
\int x e^x\,dx = x e^x - \int e^x\,dx
\]

Resultado:
\[
\boxed{x e^x - e^x + C}
\]

\subsection{Otro ejemplo}
\[
\int x\cos x\,dx
\]

\[
u=x,\quad du=dx
\]
\[
dv=\cos x\,dx,\quad v=\sin x
\]

\[
\int x\cos x\,dx = x\sin x - \int \sin x\,dx
\]

Resultado:
\[
\boxed{x\sin x + \cos x + C}
\]

\subsection{Ejercicios}
\begin{enumerate}
  \item $\displaystyle \int x\ln x\,dx$
  \item $\displaystyle \int x e^{2x}\,dx$
  \item $\displaystyle \int x\sin x\,dx$
\end{enumerate}

\section{Integrales trigonométricas utilizando identidades}

Algunas integrales trigonométricas no se resuelven directamente, sino aplicando
\textbf{identidades trigonométricas} para simplificar la expresión.

\subsection{Identidades básicas}
\[
\sin^2 x + \cos^2 x = 1
\]
\[
1+\tan^2 x=\sec^2 x
\]
\[
1+\cot^2 x=\csc^2 x
\]

También son útiles las identidades de ángulo doble:
\[
\sin^2 x=\frac{1-\cos(2x)}{2}, \qquad
\cos^2 x=\frac{1+\cos(2x)}{2}
\]

\subsection{Potencias de seno y coseno}

\subsubsection*{Caso 1: Potencia impar de seno o coseno}
Si una de las potencias es impar, se separa un factor y se usa la identidad fundamental.

\subsection{Ejemplo}
\[
\int \sin^3 x \cos x\,dx
\]

\[
\sin^3 x = \sin^2 x \sin x = (1-\cos^2 x)\sin x
\]

Sea $u=\cos x$, entonces $du=-\sin x\,dx$:
\[
-\int (1-u^2)\,du = -\left(u-\frac{u^3}{3}\right)+C
\]

Resultado:
\[
-\cos x + \frac{\cos^3 x}{3}+C
\]

\subsubsection*{Caso 2: Potencias pares de seno y coseno}
Se usan identidades de ángulo doble.

\subsection{Ejemplo}
\[
\int \sin^2 x\,dx
\]

\[
\int \frac{1-\cos(2x)}{2}\,dx
= \frac{1}{2}\left(x-\frac{\sin(2x)}{2}\right)+C
\]

\subsection{Ejercicios}
\begin{enumerate}
  \item $\displaystyle \int \cos^2 x\,dx$
  \item $\displaystyle \int \sin^3 x\,dx$
  \item $\displaystyle \int \sin^2 x \cos^2 x\,dx$
\end{enumerate}

%------------------------------------------------

\section{Sustituciones trigonométricas}

Las sustituciones trigonométricas se utilizan para integrar expresiones que contienen raíces cuadradas de la forma:
\[
\sqrt{a^2-x^2}, \quad \sqrt{a^2+x^2}, \quad \sqrt{x^2-a^2}
\]

\subsection{Sustituciones más comunes}
\begin{center}
\begin{tabular}{c c}
\toprule
Expresión & Sustitución \\
\midrule
$\sqrt{a^2-x^2}$ & $x=a\sin\theta$ \\
$\sqrt{a^2+x^2}$ & $x=a\tan\theta$ \\
$\sqrt{x^2-a^2}$ & $x=a\sec\theta$ \\
\bottomrule
\end{tabular}
\end{center}

\subsection{Ejemplo}
\[
\int \sqrt{4-x^2}\,dx
\]

Sea $x=2\sin\theta$, entonces $dx=2\cos\theta\,d\theta$:
\[
\int \sqrt{4-4\sin^2\theta}\,2\cos\theta\,d\theta
\]

\[
= \int 2\cos\theta \cdot 2\cos\theta\,d\theta
=4\int \cos^2\theta\,d\theta
\]

\[
=4\int \frac{1+\cos(2\theta)}{2}\,d\theta
\]

Resultado:
\[
2\theta+\sin(2\theta)+C
\]

Regresando a $x$:
\[
\theta=\arcsin\left(\frac{x}{2}\right)
\]

\subsection{Ejercicios}
\begin{enumerate}
  \item $\displaystyle \int \frac{1}{\sqrt{9-x^2}}\,dx$
  \item $\displaystyle \int \sqrt{x^2+4}\,dx$
  \item $\displaystyle \int \frac{1}{x\sqrt{x^2-1}}\,dx$
\end{enumerate}

\section{Fracciones parciales}

La técnica de fracciones parciales se utiliza para integrar funciones racionales, es decir,
cocientes de polinomios.

\subsection{Condición previa}
El grado del numerador debe ser \textbf{menor} que el grado del denominador.  
Si no se cumple, primero se debe realizar división de polinomios.

\subsection{Caso 1: Denominador con factores lineales distintos}
\[
\int \frac{1}{(x-1)(x+2)}\,dx
\]

Se plantea:
\[
\frac{1}{(x-1)(x+2)}=\frac{A}{x-1}+\frac{B}{x+2}
\]

Resolviendo el sistema se obtiene:
\[
A=\frac{1}{3}, \quad B=-\frac{1}{3}
\]

Por lo tanto:
\[
\int \frac{1}{(x-1)(x+2)}\,dx
=\frac{1}{3}\ln|x-1|-\frac{1}{3}\ln|x+2|+C
\]

\subsection{Caso 2: Denominador con factor cuadrático irreducible}
\[
\int \frac{x}{x^2+1}\,dx
\]

En este caso:
\[
\frac{x}{x^2+1}=\frac{1}{2}\cdot\frac{2x}{x^2+1}
\]

\[
\int \frac{x}{x^2+1}\,dx=\frac{1}{2}\ln(x^2+1)+C
\]

\subsection{Ejercicios}
\begin{enumerate}
  \item $\displaystyle \int \frac{1}{x^2-1}\,dx$
  \item $\displaystyle \int \frac{2x+1}{(x-1)(x+2)}\,dx$
  \item $\displaystyle \int \frac{1}{x^2+4x+5}\,dx$
\end{enumerate}

%------------------------------------------------

\section{Integral definida}

La integral definida representa el \textbf{área neta} bajo la curva de una función
entre dos puntos.

\subsection{Definición}
\[
\int_a^b f(x)\,dx = F(b)-F(a)
\]
donde $F'(x)=f(x)$.

Este resultado es conocido como el \textbf{Teorema Fundamental del Cálculo}.

\subsection{Interpretación geométrica}
\begin{itemize}
  \item Área sobre el eje $x$: positiva.
  \item Área bajo el eje $x$: negativa.
\end{itemize}

\subsection{Ejemplo}
\[
\int_0^2 x^2\,dx
\]

\[
\left[\frac{x^3}{3}\right]_0^2=\frac{8}{3}
\]

\subsection{Propiedades}
\[
\int_a^a f(x)\,dx=0
\]
\[
\int_a^b f(x)\,dx=-\int_b^a f(x)\,dx
\]
\[
\int_a^b (f(x)+g(x))\,dx=\int_a^b f(x)\,dx+\int_a^b g(x)\,dx
\]

\subsection{Ejercicios}
\begin{enumerate}
  \item $\displaystyle \int_1^3 (2x+1)\,dx$
  \item $\displaystyle \int_0^\pi \sin x\,dx$
  \item $\displaystyle \int_{-1}^1 x^3\,dx$
\end{enumerate}

%------------------------------------------------

\section{Área entre dos curvas}

El área entre dos curvas se calcula integrando la diferencia entre la función superior
y la función inferior.

\subsection{Fórmula general}
\[
A=\int_a^b \bigl(f(x)-g(x)\bigr)\,dx
\]
donde $f(x)\ge g(x)$ en el intervalo $[a,b]$.

\subsection{Procedimiento}
\begin{enumerate}
  \item Graficar ambas funciones.
  \item Encontrar los puntos de intersección.
  \item Identificar cuál función está arriba.
  \item Integrar la diferencia.
\end{enumerate}

\subsection{Ejemplo}
Encuentra el área entre $f(x)=x$ y $g(x)=x^2$.

Intersecciones:
\[
x=x^2 \Rightarrow x=0,1
\]

Área:
\[
A=\int_0^1 (x-x^2)\,dx
\]

\[
\left[\frac{x^2}{2}-\frac{x^3}{3}\right]_0^1=\frac{1}{6}
\]

\subsection{Ejercicios}
\begin{enumerate}
  \item Área entre $y=x^2$ y $y=2x$.
  \item Área entre $y=\sin x$ y el eje $x$ en $[0,\pi]$.
  \item Área entre $y=x^2$ y $y=1$.
\end{enumerate}

\end{document}
