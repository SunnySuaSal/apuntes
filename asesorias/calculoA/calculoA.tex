\documentclass[12pt]{article}

\usepackage[utf8]{inputenc} %Codificación de caracteres (UTF-8)
\usepackage{amsmath, amssymb, amsfonts, cancel} %Paquetes para expresiones matemáticas
\usepackage{graphicx} %Para insertar imágenes
\usepackage[spanish]{babel} %Optimiza el typesetting para documentos en español
\usepackage[letterpaper, left=1in, right=1in, top=1in, bottom=1in]{geometry}
%Para abarcar más hoja horizontalmente
\usepackage{booktabs} %Optimiza trabajar con tablas y agrega algunos comandos
\usepackage{multirow} %Para crear celdas tabulares que abarcan múltiples filas
\usepackage{float} %Mayor control sobre dónde se colocan las figuras y tablas
\usepackage{caption} %Mayor control de las captions de figuras y tablas
\usepackage{colortbl} %Para colores en tablas
\usepackage{xcolor} %Opcional, pero útil para definir colores personalizados
\definecolor{paleYellow}{RGB}{255, 255, 180} %Ejemplo con amarillo pálido
\usepackage{physics} %Notación elegante de derivadas (opcional)
\usepackage{hyperref}

\title{Asesorías Cálculo A}
\author{Jorge Alberto Suárez Saldaña}
\date{\today}

\begin{document}
\maketitle
En las presentes notas de las asesorías, recorreremos los temas listados en el temario del curso. Tomando un enfoque más práctico con una ligera explicación teórica, 
la resolución de al menos un ejercicio y la asignación de varios ejercicios de práctica.
\section{Definición intuitiva de límite}
El límite de una función en un punto describe a qué valor se acerca la función cuando x se aproxima a ese punto, aunque la función no esté definida allí.
Significa acercarce mucho a un valor sin llegar a este.
\subsection{Ejemplo}
La función $f(x)=\frac{x^2-1}{x-1}$ no se puede evaluar en $x=1$, pues se indefine (da $0/0$), pero si evaluamos en valores cercanos:
\begin{table}[H]
\centering
\begin{tabular}{cc}
$x$&$f(x)$\\
0.9&1.9\\
0.99&1.99\\
1.01&2.01\\
1.1&2.1
\end{tabular}
\end{table}
Observamos que la función se acerca a 2, por lo tanto \[ \lim_{x \to 1} f(x) = 2 \]
La definición intuitiva formal es la siguiente: \[ \lim_{x \to a} f(x)=L \]
Que se lee como: El límite de $f(x)$ cuando x tiende a $a$ es $L$.
\subsection{Ejercicios (tabulación)}
\begin{enumerate}
  \item \[ \lim_{x \to 2} (x^2+1) \]
  \item \[ \lim_{x \to 0} \frac{\sin(x)}{x} \]
  \item \[ \lim_{x \to 1} \frac{|x-1|}{x-1} \]
\end{enumerate}
\section{Análisis gráfico de límites}
Podemos examinar los límites de una función si contamos con la gráfica de dicha función, 
especialmente si utilizamos un graficador digital como \textit{Desmos} o \textit{GeoGebra}.
El procedimiento es el siguiente:
\begin{enumerate}
  \item Observa la gráfica.
  \item Observa qué valor toma $f(x)$ cuando te acercas por izquierda y popr derecha.
  \item Si \underline{ambos} coinciden, el límite existe.
\end{enumerate}
Casos importantes en gráficas:
\begin{enumerate}
  \item \textbf{Hueco:} La gráfica se acerca a un punto, pero no está definida allí. Entonces sí hay límite. Ejemplo en la Figura \ref{fig:limitePunto}.
  \item \textbf{Salto:} Valores distintos por izquierda y derecha. Entonces el límite no existe. Ejemplo en la Figura \ref{fig:limiteSalto}.
  \item \textbf{Asíntota vertical:} La gráfica se dispara. Entonces el límite es infinito. Ejemplo en la Figura \ref{fig:limiteAsintota}.
\end{enumerate}
\begin{figure}
  \begin{center}
    \includegraphics[width=0.6\textwidth]{assets/limitePunto.jpeg}
  \end{center}
  \caption{Ejemplo de límite hueco.}\label{fig:limitePunto}
\end{figure}
\begin{figure}
  \begin{center}
    \includegraphics[width=0.6\textwidth]{assets/limiteSalto.jpeg}
  \end{center}
  \caption{Ejemplo de límite con salto.}\label{fig:limiteSalto}
\end{figure}
\begin{figure}
  \begin{center}
    \includegraphics[width=0.6\textwidth]{assets/limiteAsintota.jpeg}
  \end{center}
  \caption{Ejemplo de límite con asíntota.}\label{fig:limiteAsintota}
\end{figure}



\end{document}
