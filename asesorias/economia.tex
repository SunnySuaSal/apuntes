\documentclass[12pt]{article}

\usepackage[utf8]{inputenc} %Codificación de caracteres (UTF-8)
\usepackage{amsmath, amssymb, amsfonts} %Paquetes para expresiones matemáticas
\usepackage{graphicx} %Para insertar imágenes
\usepackage[spanish]{babel} %Optimiza el typesetting para documentos en español
\usepackage[letterpaper, left=1in, right=1in, top=1in, bottom=1in]{geometry}
%Para abarcar más hoja horizontalmente
\usepackage{booktabs} %Optimiza trabajar con tablas y agrega algunos comandos
\usepackage{multirow} %Para crear celdas tabulares que abarcan múltiples filas
\usepackage{float} %Mayor control sobre dónde se colocan las figuras y tablas
\usepackage{caption} %Mayor control de las captions de figuras y tablas
\usepackage{colortbl} %Para colores en tablas
\usepackage{xcolor} %Opcional, pero útil para definir colores personalizados
\definecolor{paleYellow}{RGB}{255, 255, 180} %Ejemplo con amarillo pálido
\usepackage{physics} %Notación elegante de derivadas (opcional)
\usepackage{hyperref}

\title{Asesorías Economía}
\author{Jorge Alberto Suárez Saldaña}
\date{\today}

\begin{document}
\maketitle
Podemos comenzar resolviendo los ejercicios de tus exámenes pasados y analizar en ellos los temas que abarcan los incisos.
\section{Dominio y rango de una función.}
La primera pregunta del primer examen dice: \textit{Determina el dominio de la función $f(x) = \sqrt{3x^2-2x-1}$ y expresa tu resultado con intervalos.}
\subsection{Dominio de una función.}
Te pregunta por el \textbf{dominio} de una función, es decir, dónde está definida.
En este caso, las raíces cuadradas sólo están definidas cuando el radicando es $\geq 0$.
\begin{gather*}
  3x^2-2x-1 \geq 0\\
  3x^2-2x-1 = 0\\
  (3x+1)(x-1) = 0\\
  x_1 = -\frac{1}{3} \ x_2 = 1\\
  \text{Entonces tenemos los intervalos}\\
  (-\infty , -\frac{1}{3}), (-\frac{1}{3}, 1), (1, \infty)
\end{gather*}
Tomas un valor de prueba para cada intervalo, evalúas el sgno resultante de la función y tomas los que cumplen con la desigualdad.

\begin{table}[H]
  \centering
  \begin{tabular}{ccccc}
    Intervalo&Valor&$3x+1$&$x-1$&Producto\\
    $(-\infty , -\frac{1}{3})$&$-1$&$-$&$-$&$+$\\
    $(-\frac{1}{3}, 1)$&$0$&$+$&$-$&$-$\\
    $(1, \infty)$&$2$&$+$&$+$&$+$
  \end{tabular}
\end{table}

\textbf{Resultado:} El dominio se encuentra comprendido entre los intervalos \[ (-\infty , -\tfrac{1}{3}) \cup (1,\infty) \]
\subsection{Ejercicios}
\begin{enumerate}
  \item $\sqrt{5x-x^2}$
  \item $\sqrt{\frac{x-1}{x+2}}$
  \item $\sqrt{4x^2-9}$
\end{enumerate}

Por supuesto, la función racional es solamente un tipo de función, seguramente en otros examen te pidan encontrar el dominio de otro tipo de función, por lo que 
vale la pena estudiar los limites de otros tipos de función:
\begin{itemize}
  \item \textbf{Polinomios:} Definidos para todo número real.\\Ejemplos: $f(x)=x^3-4x+1$, $f(x)=7x^2+\sqrt{2}$\\Dominio: $(-\infty, \infty)$.
  \item \textbf{Funciones racionales:} El denominador \underline{no} puede ser cero.\\Ejemplo: $f(x)=\frac{1}{x-2}$. $x \neq 0$\\[0.5em] Dominio: $(-\infty, 2) \cup (2, \infty)$
  \item \textbf{Raíces de índice impar:} No hay restricción.\\Ejemplo: $\sqrt[3]{x-7}$.\\Dominio: $(-\infty, \infty)$.
  \item \textbf{Logaritmos:} El argumento del logaritmo debe ser $>0$.\\Ejemplo: $\ln(x-3) \rightarrow x-3 > 0 \rightarrow x > 3$.
\end{itemize}
\subsection{Ejercicios}
\begin{enumerate}
  \item $x^4-3x^2+5$
  \item $2x^5-x+9$
  \item $\frac{3x}{x+1}$
  \item $\frac{x^2-1}{x^2-4}$
  \item $\sqrt{2x+3}$
  \item $\sqrt{4-x^2}$
  \item $\sqrt[3]{x^2-1}$
  \item $\sqrt[5]{3x+2}$
  \item $\log(x+5)$
  \item $\ln(4-x^2)$
\end{enumerate}

\section{Función inversa}
La segunda pregunta del examen plantea: \textit{Para $f(x)=\frac{3-2x}{2x-1}$, determina $f^{-1}(x)$ y comprueba que $f^{-1}(f(x))=x$. 
Además escribe para qué conjunto es válida dicha composición.}
El procedimiento es sencillo:
\begin{enumerate}
  \item Escribir $y=f(x)$
    \[
      y=\frac{3-2x}{2x-1}
    \]
  \item Despejar $x$
    \[
      y(2x-1)=3-2x\\
      2xy-y=3-2x\\
      2xy+2x=3+y\\
      x(2y+2)=3+y\\
      x=\frac{3+y}{2y+2}
    \]
  \item intercambiar variables
    \[
      f^{-1}(x)=\frac{x+3}{2x+2}
    \]
\end{enumerate}

Verificación (por sustitución directa)

\[ \frac{\frac{3-2x}{2x-1}+3}{2(\frac{3-2x}{2x-1})+2}=\frac{\frac{3-2x}{2x-1}+\frac{3(2x-1)}{2x-1}}{\frac{6-4x}{2x-1}+\frac{2(2x-1)}{2x-1}} \]
\[ \frac{\frac{3-2x+6x-3}{2x-1}}{\frac{6-4x+4x-2}{2x-1}}=\frac{\frac{4x}{2x-1}}{\frac{4}{2x-1}} \]
Por regla del sándwich, simplificamos:
\[ \frac{4x(2x-1)}{4(2x-1)} = x \]

\subsection{Dominio de la composición}
La función $f(x)$ no está definida cuando el denominador $2x-1=0$, por despeje $x \neq \frac{1}{2}$.
La función inversa $f^{-1}(x)$ no está definida cuando el denominador $2x+2=0$, por despeje $x \neq -1$
Entonces, el dominio de la composición de función y función inversa es \[ \mathbb{R} \setminus \left\{\frac{1}{2}, -1\right\}\]

\subsection{Ejercicios}
\begin{enumerate}
  \item \[f(x)=\frac{x+1}{x-3}\]
  \item \[f(x)=\frac{2x-5}{3}\]
\end{enumerate}

\end{document}
