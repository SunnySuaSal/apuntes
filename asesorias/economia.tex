\documentclass[12pt]{article}

\usepackage[utf8]{inputenc} %Codificación de caracteres (UTF-8)
\usepackage{amsmath, amssymb, amsfonts} %Paquetes para expresiones matemáticas
\usepackage{graphicx} %Para insertar imágenes
\usepackage[spanish]{babel} %Optimiza el typesetting para documentos en español
\usepackage[letterpaper, left=1in, right=1in, top=1in, bottom=1in]{geometry}
%Para abarcar más hoja horizontalmente
\usepackage{booktabs} %Optimiza trabajar con tablas y agrega algunos comandos
\usepackage{multirow} %Para crear celdas tabulares que abarcan múltiples filas
\usepackage{float} %Mayor control sobre dónde se colocan las figuras y tablas
\usepackage{caption} %Mayor control de las captions de figuras y tablas
\usepackage{colortbl} %Para colores en tablas
\usepackage{xcolor} %Opcional, pero útil para definir colores personalizados
\definecolor{paleYellow}{RGB}{255, 255, 180} %Ejemplo con amarillo pálido
\usepackage{physics} %Notación elegante de derivadas (opcional)
\usepackage{hyperref}

\title{Asesorias Economia}

