\documentclass[12pt]{article}

\usepackage[utf8]{inputenc} %Codificación de caracteres (UTF-8)
\usepackage{amsmath, amssymb, amsfonts, cancel} %Paquetes para expresiones matemáticas
\usepackage{graphicx} %Para insertar imágenes
\usepackage[spanish]{babel} %Optimiza el typesetting para documentos en español
\usepackage[letterpaper, left=1in, right=1in, top=1in, bottom=1in]{geometry}
%Para abarcar más hoja horizontalmente
\usepackage{booktabs} %Optimiza trabajar con tablas y agrega algunos comandos
\usepackage{multirow} %Para crear celdas tabulares que abarcan múltiples filas
\usepackage{float} %Mayor control sobre dónde se colocan las figuras y tablas
\usepackage{caption} %Mayor control de las captions de figuras y tablas
\usepackage{colortbl} %Para colores en tablas
\usepackage{xcolor} %Opcional, pero útil para definir colores personalizados
\definecolor{paleYellow}{RGB}{255, 255, 180} %Ejemplo con amarillo pálido
\usepackage{physics} %Notación elegante de derivadas (opcional)
\usepackage{hyperref}

\title{Asesorías Economía}
\author{Jorge Alberto Suárez Saldaña}
\date{\today}

\begin{document}
\maketitle
Podemos comenzar resolviendo los ejercicios de tus exámenes pasados y analizar en ellos los temas que abarcan los incisos.
\section{Dominio y rango de una función.}
La primera pregunta del primer examen dice: \textit{Determina el dominio de la función $f(x) = \sqrt{3x^2-2x-1}$ y expresa tu resultado con intervalos.}
\subsection{Dominio de una función.}
Te pregunta por el \textbf{dominio} de una función, es decir, dónde está definida.
En este caso, las raíces cuadradas sólo están definidas cuando el radicando es $\geq 0$.
\begin{gather*}
  3x^2-2x-1 \geq 0\\
  3x^2-2x-1 = 0\\
  (3x+1)(x-1) = 0\\
  x_1 = -\frac{1}{3} \ x_2 = 1\\
  \text{Entonces tenemos los intervalos}\\
  (-\infty , -\frac{1}{3}), (-\frac{1}{3}, 1), (1, \infty)
\end{gather*}
Tomas un valor de prueba para cada intervalo, evalúas el sgno resultante de la función y tomas los que cumplen con la desigualdad.

\begin{table}[H]
  \centering
  \begin{tabular}{ccccc}
    Intervalo&Valor&$3x+1$&$x-1$&Producto\\
    $(-\infty , -\frac{1}{3})$&$-1$&$-$&$-$&$+$\\
    $(-\frac{1}{3}, 1)$&$0$&$+$&$-$&$-$\\
    $(1, \infty)$&$2$&$+$&$+$&$+$
  \end{tabular}
\end{table}

\textbf{Resultado:} El dominio se encuentra comprendido entre los intervalos \[ (-\infty , -\tfrac{1}{3}) \cup (1,\infty) \]
\subsection{Ejercicios}
\begin{enumerate}
  \item $\sqrt{5x-x^2}$
  \item $\sqrt{\frac{x-1}{x+2}}$
  \item $\sqrt{4x^2-9}$
\end{enumerate}

Por supuesto, la función racional es solamente un tipo de función, seguramente en otros examen te pidan encontrar el dominio de otro tipo de función, por lo que 
vale la pena estudiar los limites de otros tipos de función:
\begin{itemize}
  \item \textbf{Polinomios:} Definidos para todo número real.\\Ejemplos: $f(x)=x^3-4x+1$, $f(x)=7x^2+\sqrt{2}$\\Dominio: $(-\infty, \infty)$.
  \item \textbf{Funciones racionales:} El denominador \underline{no} puede ser cero.\\Ejemplo: $f(x)=\frac{1}{x-2}$. $x \neq 0$\\[0.5em] Dominio: $(-\infty, 2) \cup (2, \infty)$
  \item \textbf{Raíces de índice impar:} No hay restricción.\\Ejemplo: $\sqrt[3]{x-7}$.\\Dominio: $(-\infty, \infty)$.
  \item \textbf{Logaritmos:} El argumento del logaritmo debe ser $>0$.\\Ejemplo: $\ln(x-3) \rightarrow x-3 > 0 \rightarrow x > 3$.
\end{itemize}
\subsection{Ejercicios}
\begin{enumerate}
  \item $x^4-3x^2+5$
  \item $2x^5-x+9$
  \item $\frac{3x}{x+1}$
  \item $\frac{x^2-1}{x^2-4}$
  \item $\sqrt{2x+3}$
  \item $\sqrt{4-x^2}$
  \item $\sqrt[3]{x^2-1}$
  \item $\sqrt[5]{3x+2}$
  \item $\log(x+5)$
  \item $\ln(4-x^2)$
\end{enumerate}

\section{Función inversa}
La segunda pregunta del examen plantea: \textit{Para $f(x)=\frac{3-2x}{2x-1}$, determina $f^{-1}(x)$ y comprueba que $f^{-1}(f(x))=x$. 
Además escribe para qué conjunto es válida dicha composición.}
El procedimiento es sencillo:
\begin{enumerate}
  \item Escribir $y=f(x)$
    \[
      y=\frac{3-2x}{2x-1}
    \]
  \item Despejar $x$
    \[
      y(2x-1)=3-2x\\
      2xy-y=3-2x\\
      2xy+2x=3+y\\
      x(2y+2)=3+y\\
      x=\frac{3+y}{2y+2}
    \]
  \item intercambiar variables
    \[
      f^{-1}(x)=\frac{x+3}{2x+2}
    \]
\end{enumerate}

\paragraph{Verificación (por sustitución directa)}

\begin{align*}
\frac{\dfrac{3-2x}{2x-1}+3}{2\left(\dfrac{3-2x}{2x-1}\right)+2}
&=
\frac{\dfrac{3-2x}{2x-1}+\dfrac{3(2x-1)}{2x-1}}
     {\dfrac{6-4x}{2x-1}+\dfrac{2(2x-1)}{2x-1}}
\\[1em]
&=
\frac{\dfrac{3-2x+6x-3}{2x-1}}
     {\dfrac{6-4x+4x-2}{2x-1}}
\\[1em]
&=
\frac{\dfrac{4x}{2x-1}}
     {\dfrac{4}{2x-1}}
\end{align*}

\medskip
Por la \emph{regla del sándwich}, simplificamos:

\[
\frac{\cancel{4}x\cancel{(2x-1)}}{\cancel{4}\cancel{(2x-1)}} = x
\]

\subsection{Dominio de la composición}
La función $f(x)$ no está definida cuando el denominador $2x-1=0$, por despeje $x \neq \frac{1}{2}$.
La función inversa $f^{-1}(x)$ no está definida cuando el denominador $2x+2=0$, por despeje $x \neq -1$
Entonces, el dominio de la composición de función y función inversa es \[ \mathbb{R} \setminus \left\{\frac{1}{2}, -1\right\}\]

\subsection{Ejercicios}
\begin{enumerate}
  \item \[f(x)=\frac{x+1}{x-3}\]
  \item \[f(x)=\frac{2x-5}{3}\]
\end{enumerate}

\section{Continuidad de una función}
La tercera pregunta del examen plantea: \textit{Para la función $f(x)=\frac{2x^2-x-1}{x^2-3x+2}$ determina los puntos donde $f(x)$ es discontinua y determina
el tipo de discontinuidad en cada punto.}

Este problema nos pregunta por la continuidad (o en su defecto, discontinuidad) de una función. 
Para encontrar sichos puntos de discontinuidad, el procedimiento es el siguiente:

\begin{enumerate}
  \item Factorizar
    \[ 2x^2-x-1 = (2x+1)(x-1) \]
    \[ x^2-3x+2 = (x-1)(x-2) \]
  \item Encontrar puntos problemáticos. Aquellos en los que el denomidador $=0$
    En este caso, es en los valores $x=1$ y $x=2$
  \item Simplificar
    \[ 
      f(x) = \frac{(2x+1)(\cancel{x-1})}{(\cancel{x-1})(x-2)}
    \]
    \[
      \frac{2x-1}{x-2}, x \neq 1,2
    \]
  \item Clasificación de los tipos de discontinuidad

    En $x=1$ el factor se cancela, la función no está definida pero existe el límite.
    A esto se le conoce como \textbf{discontinuidad removible} o hueco.

    En $x=2$ el factor no se cancela, por lo que se comporta como una \textbf{asíntota vertical}, también conocida como \textbf{discontinuidad infinita}.
\end{enumerate}

\subsection{Ejercicios}
\begin{enumerate}
  \item \[ \frac{x^2-4}{x-2} \]
  \item \[ \frac{x^2+1}{x^2-1} \]
  \item \[ \frac{x^2-9}{x^2-3x} \]
\end{enumerate}

\section{Crecimiento, decrecimiento y concavidad}
La quinta pregunta del examen plantea: \textit{Considera la función
\[
f(x)=2x^3-9x^2+12x-6
\]
Determina la primera y segunda derivada, los intervalos donde la función es creciente o decreciente y los intervalos donde es convexa o cóncava.}

\subsection{Primera y segunda derivada}
Derivamos término a término:

\[
f'(x)=6x^2-18x+12
\]

Factorizando:

\[
f'(x)=6(x-1)(x-2)
\]

La segunda derivada es:

\[
f''(x)=12x-18
\]

---

\subsection{Crecimiento y decrecimiento}
Los intervalos de crecimiento y decrecimiento se obtienen analizando el signo de la primera derivada.

Resolvemos:
\[
f'(x)=0 \Rightarrow x=1,\; x=2
\]

Estos puntos dividen la recta real en los intervalos:
\[
(-\infty,1),\quad (1,2),\quad (2,\infty)
\]

\begin{table}[H]
  \centering
  \begin{tabular}{ccc}
    Intervalo & Signo de $f'(x)$ & Comportamiento\\
    $(-\infty,1)$ & $+$ & Creciente\\
    $(1,2)$ & $-$ & Decreciente\\
    $(2,\infty)$ & $+$ & Creciente
  \end{tabular}
\end{table}

\textbf{Resultado:}
\[
\text{Creciente en } (-\infty,1)\cup(2,\infty)
\]
\[
\text{Decreciente en } (1,2)
\]

---

\subsection{Concavidad y convexidad}
La concavidad se determina analizando el signo de la segunda derivada.

\[
f''(x)=12x-18=0 \Rightarrow x=\frac{3}{2}
\]

Dividimos la recta real:

\[
(-\infty,\tfrac{3}{2}),\quad (\tfrac{3}{2},\infty)
\]

\begin{table}[H]
  \centering
  \begin{tabular}{ccc}
    Intervalo & Signo de $f''(x)$ & Forma\\
    $(-\infty,\tfrac{3}{2})$ & $-$ & Cóncava\\
    $(\tfrac{3}{2},\infty)$ & $+$ & Convexa
  \end{tabular}
\end{table}

\textbf{Resultado:}
\[
\text{Cóncava en } (-\infty,\tfrac{3}{2})
\]
\[
\text{Convexa en } (\tfrac{3}{2},\infty)
\]

\subsection{Ejercicios}
\begin{enumerate}
  \item $f(x)=x^3-3x^2+2$
  \item $f(x)=-x^2+4x-1$
  \item $f(x)=x^4-4x^2$
\end{enumerate}

\section{Máximos y mínimos locales}

La sexta pregunta del examen plantea: \textit{Determina los máximos y mínimos locales de la función
$f(x)=\frac{1}{2}x^2+\frac{1}{x}$ usando el criterio de la segunda derivada.}

\subsection{Dominio de la función}

Antes de aplicar criterios de derivadas, es importante determinar el dominio de la función.
El término $\frac{1}{x}$ no está definido en $x=0$, por lo que:

\[
\text{Dom}(f)=\mathbb{R}\setminus\{0\}
\]

---

\subsection{Primera derivada}

Derivamos término a término:

\[
f'(x)=\dv{}{x}\left(\frac{1}{2}x^2\right)+\dv{}{x}\left(\frac{1}{x}\right)
\]

\[
f'(x)=x-\frac{1}{x^2}
\]

---

\subsection{Puntos críticos}

Los puntos críticos se obtienen resolviendo:

\[
f'(x)=0
\]

\[
x-\frac{1}{x^2}=0
\]

Multiplicamos toda la ecuación por $x^2$ (válido ya que $x\neq 0$):

\[
x^3-1=0
\]

\[
x^3=1 \quad \Rightarrow \quad x=1
\]

Por lo tanto, el único punto crítico es $x=1$.

---

\subsection{Segunda derivada}

Derivamos nuevamente:

\[
f''(x)=\dv{}{x}\left(x-\frac{1}{x^2}\right)
\]

\[
f''(x)=1+\frac{2}{x^3}
\]

---

\subsection{Criterio de la segunda derivada}

Evaluamos la segunda derivada en el punto crítico $x=1$:

\[
f''(1)=1+\frac{2}{1^3}=3
\]

Como $f''(1)>0$, la función es \textbf{cóncava hacia arriba} en ese punto, por lo tanto existe un
\textbf{mínimo local} en $x=1$.

---

\subsection{Valor del mínimo local}

Evaluamos la función en $x=1$:

\[
f(1)=\frac{1}{2}(1)^2+\frac{1}{1}=\frac{1}{2}+1=\frac{3}{2}
\]

---

\subsection{Conclusión}

La función presenta:
\begin{itemize}
  \item Un \textbf{mínimo local} en el punto $\left(1,\frac{3}{2}\right)$.
  \item No presenta máximos locales.
\end{itemize}

\subsection{Ejercicios}

Determina los máximos y mínimos locales de las siguientes funciones utilizando el
\textbf{criterio de la segunda derivada}.  
En cada caso:
\begin{itemize}
  \item Encuentra el dominio.
  \item Calcula la primera y segunda derivada.
  \item Determina y clasifica los puntos críticos.
\end{itemize}

\begin{enumerate}
  \item \[
    f(x)=x^3-3x^2+2
  \]

  \item \[
    f(x)=x^4-4x^2
  \]

  \item \[
    f(x)=\frac{1}{3}x^3-x
  \]

  \item \[
    f(x)=x^2+\frac{4}{x}
  \]

  \item \[
    f(x)=\ln(x)+x^2 \quad (x>0)
  \]

  \item \[
    f(x)=\frac{x^2+1}{x}
  \]

  \item \[
    f(x)=x^2-2x+\frac{1}{x}
  \]

  \item \[
    f(x)=e^x-x
  \]

 \end{enumerate}

\paragraph{Comentario}
No todos los puntos críticos pueden clasificarse con el criterio de la segunda derivada.
En caso de que $f''(x_0)=0$, se debe recurrir al análisis del signo de la primera derivada
o al estudio gráfico de la función.

\section{Integrales definidas}

La séptima pregunta del examen plantea: \textit{Calcula las siguientes integrales. Escribe todos los pasos que seguiste para obtener el resultado final.}

\subsection*{Inciso (a)}

\[
\int_{-1}^{1} (2-7x+6x^2)\,dx
\]

\paragraph{Paso 1: Integrar término a término}

\[
\int (2-7x+6x^2)\,dx
=
\int 2\,dx
-
\int 7x\,dx
+
\int 6x^2\,dx
\]

\[
= 2x - \frac{7}{2}x^2 + 2x^3
\]

\paragraph{Paso 2: Evaluar en los límites}

\[
\left[2x - \frac{7}{2}x^2 + 2x^3\right]_{-1}^{1}
=
\left(2(1)-\frac{7}{2}(1)^2+2(1)^3\right)
-
\left(2(-1)-\frac{7}{2}(-1)^2+2(-1)^3\right)
\]

\paragraph{Paso 3: Calcular cada evaluación}

\[
= \left(2-\frac{7}{2}+2\right)
-
\left(-2-\frac{7}{2}-2\right)
\]

\[
= \frac{1}{2} - \left(-\frac{15}{2}\right)
\]

\[
= \frac{16}{2}=8
\]

\paragraph{Resultado}

\[
\boxed{8}
\]

---

\subsection*{Inciso (b)}

\[
\int_{1}^{2} 2(2x-2)^7\,dx
\]

\paragraph{Paso 1: Sustitución}

Sea:
\[
u = 2x-2 \quad \Rightarrow \quad du = 2\,dx
\]

La integral se transforma en:

\[
\int u^7\,du
\]

\paragraph{Paso 2: Cambiar los límites}

Cuando $x=1$:
\[
u = 2(1)-2 = 0
\]

Cuando $x=2$:
\[
u = 2(2)-2 = 2
\]

Entonces:

\[
\int_{0}^{2} u^7\,du
\]

\paragraph{Paso 3: Integrar}

\[
\int u^7\,du = \frac{u^8}{8}
\]

\paragraph{Paso 4: Evaluar en los límites}

\[
\left[\frac{u^8}{8}\right]_{0}^{2}
=
\frac{2^8}{8}-0
=
\frac{256}{8}
=32
\]

\paragraph{Resultado}

\[
\boxed{32}
\]

\subsection{Ejercicios}

Calcula las siguientes integrales definidas.  
En cada caso, escribe todos los pasos y justifica el método utilizado.

\begin{enumerate}
  \item \[
    \int_{0}^{3} (4x^2-6x+1)\,dx
  \]

  \item \[
    \int_{-2}^{2} (5x^3-4x)\,dx
  \]
  \textit{(Sugerencia: observa la simetría de la función.)}

  \item \[
    \int_{1}^{4} (3x^2-2x)\,dx
  \]

  \item \[
    \int_{0}^{1} (6-8x+2x^2)\,dx
  \]

  \item \[
    \int_{1}^{2} 3(3x-1)^2\,dx
  \]

  \item \[
    \int_{0}^{2} 4(x+1)^3\,dx
  \]

  \item \[
    \int_{-1}^{1} (x^4-x^2)\,dx
  \]

  \item \[
    \int_{2}^{5} (2x-3)^5\,dx
  \]

  \item \[
    \int_{1}^{e} \frac{1}{x}\,dx
  \]

  \item \[
    \int_{0}^{\pi} \sin x \,dx
  \]
\end{enumerate}

\paragraph{Comentario}
\begin{itemize}
  \item En integrales de polinomios se integra término a término.
  \item Cuando aparece una expresión del tipo $(ax+b)^n$, conviene usar sustitución.
  \item En funciones pares o impares, el intervalo simétrico puede simplificar el cálculo.
\end{itemize}


\end{document}
